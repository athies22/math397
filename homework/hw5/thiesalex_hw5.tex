%!TEX output_directory = temp
\documentclass[letterpaper, 12pt]{amsart}
	%%%%%%%%%%%%%%%%%%%%%%%%%%%%%%%%%%%%%%%%%%%%%%%%%%%%%%%%%%%%%%%%%%%%%%%%%%%%%%
	%%%%%%%%%%%%%%%%%%%%%%%%%%%% boilerplate packages %%%%%%%%%%%%%%%%%%%%%%%%%%%%
		\usepackage[margin=1.5in]{geometry}
		\usepackage{amsmath,amssymb,amsthm}
		\usepackage{marvosym}
		\usepackage[mathscr]{euscript}
		\usepackage{enumerate}
		\usepackage{graphicx}
		\usepackage{mathrsfs}
		\usepackage{color}
		\usepackage{hyperref}
		\usepackage{verbatim}
		\usepackage{stmaryrd}

	%%%%%%%%%%%%%%%%%%%%%%%%%%%%%%%%%%%%%%%%%%%%%%%%%%%%%%%%%%%%%%%%%%%%%%%%%%%%%%
	%%%%%%%%%%%%%%%%%%%%%%%%%%%%% rename the abstract %%%%%%%%%%%%%%%%%%%%%%%%%%%%
		% \renewcommand{\abstractname}{Introduction}

	%%%%%%%%%%%%%%%%%%%%%%%%%%%%%%%%%%%%%%%%%%%%%%%%%%%%%%%%%%%%%%%%%%%%%%%%%%%%%%
	%%%%%%%%%%%%%%%%%%%%%%%%%%%%%%%%%%%%% sets %%%%%%%%%%%%%%%%%%%%%%%%%%%%%%%%%%%
		\DeclareMathOperator{\N}{\mathbb{N}}				% natural numbers
		\DeclareMathOperator{\Z}{\mathbb{Z}}				% integers
		\DeclareMathOperator{\Zp}{\mathbb{Z}^{+}}			% positive integers
		\DeclareMathOperator{\Q}{\mathbb{Q}}				% rationals
		\DeclareMathOperator{\Qc}{\mathbb{Q}^{c}}			% irrationals
		\DeclareMathOperator{\R}{\mathbb{R}}				% reals
		\DeclareMathOperator{\F}{\mathbb{F}}				% a field
		\DeclareMathOperator{\C}{\mathbb{C}}				% complex numbers
		\DeclareMathOperator{\Cnon}{\mathbb{C}^{\times}}	% nonzero complex numbers
		\DeclareMathOperator{\Pcal}{\mathcal{P}}			% powerset, or set of polynomials
		\DeclareMathOperator{\Ell}{\mathscr{L}}				% set of linear maps, or linear operator

	%%%%%%%%%%%%%%%%%%%%%%%%%%%%%%%%%%%%%%%%%%%%%%%%%%%%%%%%%%%%%%%%%%%%%%%%%%%%%%
	%%%%%%%%%%%%%%%%%%%%%%%%%%%%%% use pretty letters %%%%%%%%%%%%%%%%%%%%%%%%%%%%
		\DeclareMathOperator{\ep}{\varepsilon}				% epsilons
		\DeclareMathOperator{\ph}{\varphi}					% phis

	%%%%%%%%%%%%%%%%%%%%%%%%%%%%%%%%%%%%%%%%%%%%%%%%%%%%%%%%%%%%%%%%%%%%%%%%%%%%%%
	%%%%%%%%%%%%%%%%%%%%%%%%%%%%%%%%%%% algebra %%%%%%%%%%%%%%%%%%%%%%%%%%%%%%%%%%
		\renewcommand{\null}{\text{null }}					% null space
		\DeclareMathOperator{\range}{\text{range }}			% range
		\newcommand{\bmat}[1]{{\mathbf{#1}}}				% bold matrix
		\newcommand{\bvec}[1]{{\vec{\mathbf{#1}}}}			% bold vector
		\DeclareMathOperator{\ind}{\perp\!\!\!\perp}		% perpendicular, orthogonal
		\DeclareMathOperator{\ord}{\text{ord}}				% order of a structure
		\DeclareMathOperator{\Log}{Log}						% logarithm
		\DeclareMathOperator{\Span}{Span}					% span
		\newcommand{\pid}[1]{\langle #1 \rangle}			% bracket notation, used for 
															% ideals or inner products
		\newcommand{\norm}[1]{\mid \!\!#1 \!\!\mid}			%\norm{x} gives |x|

		% fatdot notation
		\makeatletter
			\newcommand*\fatdot{\mathpalette\fatdot@{.5}}
			\newcommand*\fatdot@[2]{\mathbin{\vcenter{\hbox{\scalebox{#2}{$\m@th#1\bullet$}}}}}
		\makeatother

	%%%%%%%%%%%%%%%%%%%%%%%%%%%%%%%%%%%%%%%%%%%%%%%%%%%%%%%%%%%%%%%%%%%%%%%%%%%%%%
	%%%%%%%%%%%%%%%%%%%%%%%%%%% probability & statistics %%%%%%%%%%%%%%%%%%%%%%%%%
		\renewcommand{\Pr}{\mathbb{P}}						% probability
		\DeclareMathOperator{\E}{\mathbb{E}}				% expectation
		\DeclareMathOperator{\var}{\rm Var}					% variance
		\DeclareMathOperator{\sd}{\rm SD}					% standard deviation
		\DeclareMathOperator{\cov}{\rm Cov}					% covariance
		\DeclareMathOperator{\SE}{\rm SE}					% standard error
		\DeclareMathOperator{\ssreg}{{\rm SS}_{{\rm Reg}}}	% sum of squared regression
		\DeclareMathOperator{\ssr}{{\rm SS}_{{\rm Res}}}	% sum of squared residuals
		\DeclareMathOperator{\sst}{{\rm SS}_{{\rm Tot}}}	% total sum of squares

	%%%%%%%%%%%%%%%%%%%%%%%%%%%%%%%%%%%%%%%%%%%%%%%%%%%%%%%%%%%%%%%%%%%%%%%%%%%%%%
	%%%%%%%%%%%%%%%%%%%%%%%%%%%%%%% number theory %%%%%%%%%%%%%%%%%%%%%%%%%%%%%%%%
		\renewcommand{\mod}[1]{\ (\mathrm{mod}\ #1)}		% congruences

	%%%%%%%%%%%%%%%%%%%%%%%%%%%%%%%%%%%%%%%%%%%%%%%%%%%%%%%%%%%%%%%%%%%%%%%%%%%%%%
	%%%%%%%%%%%%%%%%%%%%%%%%%%%% theorem environments %%%%%%%%%%%%%%%%%%%%%%%%%%%%
		% Some theorem-like environments, all numbered together starting at 1
		% in each section.

		\newtheorem{thm}{Theorem}[section]					% The default \theoremstyle is 
		\newtheorem{defn}[thm]{Definition}					% bold headings and italic body text.
		\newtheorem{prop}[thm]{Proposition}
		\newtheorem{claim}[thm]{Claim}
		\newtheorem{cor}[thm]{Corollary}
		\newtheorem{lemma}[thm]{Lemma}

		\theoremstyle{definition}  							% Bold headings and Roman body text.
		\newtheorem{example}[thm]{Example}
		\newtheorem{examples}[thm]{Examples}
		\newtheorem{exercise}[thm]{Exercise}
		\newtheorem{note}[thm]{Note}
		\newtheorem{remark}[thm]{Remark}
		\newtheorem{remarks}[thm]{Remarks}
		\newtheorem{discussion}[thm]{Discussion}

		\newcommand{\dfn}{\textbf} 							% Make defined words bold.
		\newcommand{\mdfn}[1]{\dfn{\mathversion{bold}#1}} 	% Even make math symbols bold

	%%%%%%%%%%%%%%%%%%%%%%%%%%%%%%%%%%%%%%%%%%%%%%%%%%%%%%%%%%%%%%%%%%%%%%%%%%%%%%
	%%%%%%%%%%%%%%%%%%%%%%%%%%%%%%% complex numbers %%%%%%%%%%%%%%%%%%%%%%%%%%%%%%
		\DeclareMathOperator{\Arg}{Arg}						% argument of z \in \C
		\DeclareMathOperator{\re}{Re}						% real component
		\DeclareMathOperator{\im}{Im}						% imaginary component

	%%%%%%%%%%%%%%%%%%%%%%%%%%%%%%%%%%%%%%%%%%%%%%%%%%%%%%%%%%%%%%%%%%%%%%%%%%%%%%
	%%%%%%%%%%%%%%%%%%%%%%%%%%%%%%% various symbols %%%%%%%%%%%%%%%%%%%%%%%%%%%%%%
		\newcommand{\iso}{\cong}						% isometric/congruent
		\newcommand{\ra}{\rightarrow}                   % right arrow
		\newcommand{\Ra}{\Rightarrow}                   % right implies
		\newcommand{\lra}{\longrightarrow}              % long right arrow
		\newcommand{\la}{\leftarrow}                    % left arrow
		\newcommand{\La}{\Leftarrow}                    % left implies
		\newcommand{\lla}{\longleftarrow}               % long left arrow
		\newcommand{\eqra}{\llra{\sim}}                 % equivalence/isomorphism
		\newcommand{\blank}{\underbar{\ \ }}          	% An underscore, as in (__)xV
		% \newcommand{\blank}{-}                          % A hyphen, as in (-)xV
		\newcommand{\Id}{Id}                            % The identity functor
		\newcommand{\und}{\underline}
		\newcommand{\del}{\nabla}						% gradient vector

		\raggedbottom	
\begin{document}
	\title{Homework 5  -- Math 397 \\ \today}
	\author{Alex Thies \\ \href{mailto:athies@uoregon.edu}{\lowercase{athies$@$uoregon.edu}}}

	\maketitle

	\section*{Exercise 1}
		\subsection*{Part (a)}
		Use Sage to plot the polar-coordinate equaton $r = \tfrac{1}{3-\cos\theta}$. 
		Then plot the similar $r = \tfrac{1}{3-\lambda\cos\theta}$ equations for various values of $\lambda$ stretching from 1 to 7. 
		What happens to the graph? 
		What value of $\lambda$ produces an ellipse that stretches all the way out to $x = 100$?

		\begin{proof}[Solution]
		\end{proof}
		% subsection part_a (end)

		\subsection*{Part (b)}
		Now do the algebra that explains what is happening in part (a). 
		Start with $r = \tfrac{1}{3-\lambda\cos\theta}$ and rewrite that as $$1 = r(3-\lambda\cos\theta).$$
		Change to Cartesian coordinates using $r = \sqrt{x^2 + y^2}$ and $r \cos\theta = x$. 
		Do some algebra, get rid of square roots, and rearrange until you get a quadric equation. 
		What values of $\lambda$ give the different types of conics? 
		The review of quadric equations below might be helpful.\footnote{Appendix \ref{sec:review_of_quadric_equations}}

		\begin{proof}[Solution]
		\end{proof}
		% subsection part_b (end)

		\subsection*{Part (c)}
		Show that an ellipse $Ax^2 + By^2 = C$ has eccentricity $\sqrt{1 - \tfrac{B}{A}}$ if $B \leq A$ (or alternatively, $\sqrt{1 - \tfrac{A}{B}}$ when $A \leq B$).

		\begin{proof}[Solution]
		\end{proof}
		% subsection part_c (end)

		\subsection*{Part (d)}
		Using your work in (b), derive a formula for the eccentricity of the ellipse $r = \tfrac{1}{3-\lambda\cos\theta}$ in terms of $\lambda$.

		\begin{proof}[Solution]
		\end{proof}
		% subsection part_d (end)

		\subsection*{Part (e)}
		Thinking about all the above parts, fill in the blanks for the following ``morals'' (statements that are not technically true but nevertheless have some interesting content to them):
		$$ \text{an ellipse with eccentricity $e = 1$ is really a \, $\blank\blank\blank\blank\blank\blank\blank\blank\blank\blank$}$$
		$$ \text{an ellipse with eccentricity $e > 1$ is really a \, $\blank\blank\blank\blank\blank\blank\blank\blank\blank\blank$}$$

		\begin{proof}[Solution]
		\end{proof}
		% subsection part_e (end)
	% section exercise_1 (end)

	\section*{Exercise 2}
	The Runge-Kutta order 4 method is, for the most part, extremely reliable. 
	This problem will show you a place where you need to be careful, though.
		\subsection*{Part (a)}
		Solve the separable differential equation $y' = \tfrac{-x}{y}, \, y(0) = 2$ and show that the solution is $y(x) = \sqrt{4 - x^2}$. 
		What will the graph look like?

		\begin{proof}[Solution]
		\end{proof}
		% subsection part_a (end)

		\subsection*{Part (b)}
		Now try the Sage commands
		$$\verb|P = desolve_rk4(-x,y,y,ivar=x,ics=[0,2],endpoints=10,step=0.5)|$$
		$$\verb|points(P)|$$
		You will see something kind of crazy, not looking very much like what you expected. 
		Try changing the step size to 0.1; it is even crazier!

		Focus on the region where you expect the function to be defined, and blow up the graph accordingly. 
		Include the graph as your solution to this part. 
		You might have to use the ``aspect ratio'' feature (see previous homeworks) to make the graph look right.

		\begin{proof}[Solution]
		\end{proof}
		% subsection part_b (end)

		\subsection*{Part (c)}
		Numerical approximations to first-order differential equations use the slopes of the tangent lines to advance the solution from one point to the next. 
		Keeping that in mind, what do you think is going wrong with this particular differential equation that is confusing Sage? 
		Consider plotting the points on a slope field.

		\begin{proof}[Solution]
		\end{proof}
		% subsection part_c (end)
	% section exercise_2 (end)

	\section*{Exercise 3}
	This problem deals with a pendulum of length $\ell$. 
	Deriving the differential equation for a pendulum is not too difficult.\footnote{Read about it in the Acheson chapter or in the pdf posted with assignment.}
	It is $$\theta'' = -\tfrac{g}{\ell} \cdot \sin\theta,$$ which can be solved to find $\theta = A \sin(\omega t) + B \cos(\omega t)$ where $\omega = 􏰃g$. 
	In this problem we will explore the differences between our approximate model and the true pendulum. 
	For convenience, let’s just take $\ell = g$ so that $\omega = 1$. 
	The mathematics is the same no matter what the constant $g$ is, so making it 1 doesn’t hurt as far as exploring the main ideas.
	Also, one tends to think of a pendulum as a mass suspended from a string, but in this problem we will consider pendulums that go 360 degrees around. 
	In this case, the pendulum forces the mass to always be a distance $\ell$ from the origin, and so it is better to mentally replace the string with a very light rod as this is a more accurate physical model.

		\subsection*{Part (a)}
		Solve the differential equation $\theta'' = -\theta$ with $\theta(0) = 0.5$ and $\theta'(0) = 0$.

		\begin{proof}[Solution]
		\end{proof}
		% subsection part_a (end)

		\subsection*{Part (b)}
		Set z = θ′ so that for our “true pendulum” get the system of first-order equations $$\theta' = z, \hspace{5mm} z' = -\sin\theta$$
		Use the initial conditions $\theta(0) = 0.5$ and $\theta'(0) = 0$ (so the pendulum is released from a resting position). 
		The solution for our approximate model is $\theta(t) = 0.5 \cos(t)$. 
		Use Sage to compute a numerical solution to the true pendulum, and plot it on the same graph as our approximate solution. 
		Plot a time period of at least $[0, 20]$ so that you can see how the two solutions compare over time.\footnote{Note that you can use “theta” as a variable in Sage as long as you introduce it via the command var('theta').}
		If you do this correctly, you should find that the true solution and our approximate solution are pretty close. Include the graph as your solution to this part.

		\begin{proof}[Solution]
		\end{proof}
		% subsection part_b (end)

		\subsection*{Part (c)}
		Now change the initial condtions to $\theta(0) = 1$ and repeat. 
		Compare the results to $\theta(0) = 2$ and $\theta(0) = 3$. 
		Give a physical interpretation of how the true pendulum's motion when $\theta(0) = 3$ differs from our approximate model.

		\begin{proof}
			
		\end{proof}
		% subsection part_c (end)

		\subsection*{Part (d)}
		Try the initial condition $\theta(0) = 3.1415$. 
		Then try $\theta(0) = \pi$ (using ``pi'' in Sage). 
		Explain what is happening here.

		\begin{proof}[Solution]
		\end{proof}
		% subsection part_d (end)

		\subsection*{Part (e)}
		Set the initial condition as $\theta(0) = 4.5$ and look at the resulting graph. 
		Why is the sine wave shifted vertically up now?

		\begin{proof}[Solution]
		\end{proof}
		% subsection part_e (end)

		\subsection*{Part (f)}
		Now let’s change the situation, so that the pendulum starts at the bottom $(\theta(0) = 0)$ but we give it an initial velocity $\theta'(0) = 1$. 
		Compute the solution to $\theta'' = -\theta$ for this initial condition, and then plot the numerical model for the true pendulum on the same graph. 
		Use a time interval of at least $[0, 40]$ here.

		\begin{proof}[Solution]
		\end{proof}
		% subsection part_f (end)

		\subsection*{Part (g)}
		Repeat the previous part for $\theta'(0) = 2$. 
		There is a big difference in the graphs this time. 
		Explain physically what is happening with the true pendulum.

		\begin{proof}[Solution]
		\end{proof}
		% subsection part_g (end)

		\subsection*{Part (h)}
		Now try $\theta'(0) = 2.01$. 
		The graph should be very different now! 
		Sage is not doing anything wrong, and this is the true physical solution. 
		Explain what is happening here.

		\begin{proof}[Solution]
		\end{proof}
		% subsection part_h (end)
	% section exercise_3 (end)

	\appendix

	\section{Review of Quadric Equations}
	\label{sec:review_of_quadric_equations}
	Brief review of quadric equations: 
	Equations of the form $$Ax^2 + By^2 + Cx + D = 0$$ are conic sections, meaning that they give either ellipses, parabolas, or hyperbolas. 
	You can even include $Exy$ and $Fy$ terms in the equation, but I have left them out for simplicity. 
	To see what kind of conic section the equation gives, complete the square like this:
	\begin{align*}
	0 = Ax^2 + By^2 + Cx + D &= A \left( x^2 + \tfrac{C}{A}x \right) + By^2 + D, \\
	&= A \left( x + \tfrac{C}{2A}x \right)^2 + By^2 + D - \tfrac{C^2}{4A^2}, \\
	&= Au^2 + By^2 + E
	\end{align*}
	where $u = x + C$ and $E = D - \tfrac{C^2}{4A^2}$.
	The final equation $0 = Au^2 + By^2 + E$ is recognizable to be an ellipse when $A$ and $B$ have the same sign, a parabola when one of $A$ and $B$ is zero, and a hyperbola when $A$ and $B$ have opposite signs.
	% section review_of_quadric_equations (end)
\end{document}