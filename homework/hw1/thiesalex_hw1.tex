%!TEX output_directory = temp
\documentclass[12pt]{article}   % Nicer than default article style:  less flashy headings, etc.

\usepackage{amsmath,amsthm}     % Handy math stuff, theorem environments.
\usepackage{amssymb}            % Fancy math symbols.
\usepackage{euscript}           % Nice script font.
\usepackage{graphicx,enumerate}


% Don't force the bottoms of the pages to be at the same spot:
\raggedbottom


% Put sections, but not subsections, into the table of contents:
\setcounter{tocdepth}{1}
% Number sections, but not subsections.
\setcounter{secnumdepth}{1}

% Some theorem-like environments, all numbered together starting at 1 in each section.

\newcommand{\ra}{\rightarrow}
\newcommand{\R}{\mathbb{R}}
\newcommand{\lline}[1]{\overleftrightarrow{#1}}

\begin{document}

\bigskip

\begin{center}
Alex Thies \\
Math 397 \\
Homework \#1\\
\end{center}

\bigskip


\begin{enumerate}[(1)]
\item The \textit{quick brown} \textbf{fox} jumps over the \textit{lazy} \textbf{dog}.

\item {\it Italicized text}

\item {\bf Bold-face text}

\item Some normal text, {\it some italicized text}, and some {\bf bold-face text}.

\item In LaTeX, all math must be inside dollar signs.  Notice the difference between ``Let b be an element of $\R$'' and ``Let $b$ be an element of $\R$''.

\item $\displaystyle\int_{0}^1 x^7\, dx= \frac{x^8}{8} \Bigr ]^1_0 = \frac{1}{8}-\frac{0}{8}=\frac{1}{8}$.  

\item $f'(\mu)=\lim_{h \to 0} \Big [ \frac{f(\mu + h) - f(\mu)}{h} \Big ]$ and $\lim\limits_{x \to 5} \frac{\cos x}{x} =0$.


\item Let $F : \R \to \R$ be given by $f(x) = x^{3} + 3x^{2} - 1$.

\item $5<6$ and $6>2$

\item For every $x\in (0,1)$ it is true that $x^2 < x$.

\item $(f \circ g)(x) = (f(g(x))$ and $A \cap B \subseteq A$ and $X \neq Y$.
\end{enumerate}

\medskip

\hrule

\medskip

\noindent
Write your paragraph in below:


% Put your paragraph after this line

As with the previous reading, overall I enjoyed \textit{An Application of Geography to Mathematics}.
I found that the authors did a good job introducing the historical context for an integral that seems arbitrarily difficulty to the elementary student of calculus; in so doing, they weaved together a discussion about solid geometry and integral calculus that is often absent from good introductory textbooks about either subject.
Absent from this piece is a description of the options available to Mercator, and the real-world consequences of the eventual wide-spread use of his map, which I think would have made for a better overall piece of writing.\footnote{Fortunately, such a description can be found in Episode 16 of Season 2 of \textit{The West Wing}.}
\end{document}