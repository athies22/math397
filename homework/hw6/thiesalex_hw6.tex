%!TEX output_directory = temp
\documentclass[letterpaper, 12pt]{amsart}
	%%%%%%%%%%%%%%%%%%%%%%%%%%%%%%%%%%%%%%%%%%%%%%%%%%%%%%%%%%%%%%%%%%%%%%%%%%%%%%
	%%%%%%%%%%%%%%%%%%%%%%%%%%%% boilerplate packages %%%%%%%%%%%%%%%%%%%%%%%%%%%%
		\usepackage[margin=2in]{geometry}
		\usepackage{amsmath,amssymb,amsthm}
		\usepackage{marvosym}
		\usepackage[mathscr]{euscript}
		\usepackage{enumerate}
		\usepackage{graphicx}
		\usepackage{mathrsfs}
		\usepackage{color}
		\usepackage{hyperref}
		\usepackage{verbatim}
		\usepackage{stmaryrd}
		\usepackage{subfigure}

	%%%%%%%%%%%%%%%%%%%%%%%%%%%%%%%%%%%%%%%%%%%%%%%%%%%%%%%%%%%%%%%%%%%%%%%%%%%%%%
	%%%%%%%%%%%%%%%%%%%%%%%%%%%%% rename the abstract %%%%%%%%%%%%%%%%%%%%%%%%%%%%
		% \renewcommand{\abstractname}{Introduction}

	%%%%%%%%%%%%%%%%%%%%%%%%%%%%%%%%%%%%%%%%%%%%%%%%%%%%%%%%%%%%%%%%%%%%%%%%%%%%%%
	%%%%%%%%%%%%%%%%%%%%%%%%%%%%%%%%%%%%% sets %%%%%%%%%%%%%%%%%%%%%%%%%%%%%%%%%%%
		\DeclareMathOperator{\N}{\mathbb{N}}				% natural numbers
		\DeclareMathOperator{\Z}{\mathbb{Z}}				% integers
		\DeclareMathOperator{\Zp}{\mathbb{Z}^{+}}			% positive integers
		\DeclareMathOperator{\Q}{\mathbb{Q}}				% rationals
		\DeclareMathOperator{\Qc}{\mathbb{Q}^{c}}			% irrationals
		\DeclareMathOperator{\R}{\mathbb{R}}				% reals
		\DeclareMathOperator{\F}{\mathbb{F}}				% a field
		\DeclareMathOperator{\C}{\mathbb{C}}				% complex numbers
		\DeclareMathOperator{\Cnon}{\mathbb{C}^{\times}}	% nonzero complex numbers
		\DeclareMathOperator{\Pcal}{\mathcal{P}}			% powerset, or set of polynomials
		\DeclareMathOperator{\Ell}{\mathscr{L}}				% set of linear maps, or linear operator

	%%%%%%%%%%%%%%%%%%%%%%%%%%%%%%%%%%%%%%%%%%%%%%%%%%%%%%%%%%%%%%%%%%%%%%%%%%%%%%
	%%%%%%%%%%%%%%%%%%%%%%%%%%%%%% use pretty letters %%%%%%%%%%%%%%%%%%%%%%%%%%%%
		\DeclareMathOperator{\ep}{\varepsilon}				% epsilons
		\DeclareMathOperator{\ph}{\varphi}					% phis

	%%%%%%%%%%%%%%%%%%%%%%%%%%%%%%%%%%%%%%%%%%%%%%%%%%%%%%%%%%%%%%%%%%%%%%%%%%%%%%
	%%%%%%%%%%%%%%%%%%%%%%%%%%%%%%%%%%% algebra %%%%%%%%%%%%%%%%%%%%%%%%%%%%%%%%%%
		\renewcommand{\null}{\text{null }}					% null space
		\DeclareMathOperator{\range}{\text{range }}			% range
		\newcommand{\bmat}[1]{{\mathbf{#1}}}				% bold matrix
		\newcommand{\bvec}[1]{{\vec{\mathbf{#1}}}}			% bold vector
		\DeclareMathOperator{\ind}{\perp\!\!\!\perp}		% perpendicular, orthogonal
		\DeclareMathOperator{\ord}{\text{ord}}				% order of a structure
		\DeclareMathOperator{\Log}{Log}						% logarithm
		\DeclareMathOperator{\Span}{Span}					% span
		\newcommand{\pid}[1]{\langle #1 \rangle}			% bracket notation, used for 
															% ideals or inner products
		\newcommand{\norm}[1]{\mid \!\!#1 \!\!\mid}			%\norm{x} gives |x|

		% fatdot notation
		\makeatletter
			\newcommand*\fatdot{\mathpalette\fatdot@{.5}}
			\newcommand*\fatdot@[2]{\mathbin{\vcenter{\hbox{\scalebox{#2}{$\m@th#1\bullet$}}}}}
		\makeatother

	%%%%%%%%%%%%%%%%%%%%%%%%%%%%%%%%%%%%%%%%%%%%%%%%%%%%%%%%%%%%%%%%%%%%%%%%%%%%%%
	%%%%%%%%%%%%%%%%%%%%%%%%%%% probability & statistics %%%%%%%%%%%%%%%%%%%%%%%%%
		\renewcommand{\Pr}{\mathbb{P}}						% probability
		\DeclareMathOperator{\E}{\mathbb{E}}				% expectation
		\DeclareMathOperator{\var}{\rm Var}					% variance
		\DeclareMathOperator{\sd}{\rm SD}					% standard deviation
		\DeclareMathOperator{\cov}{\rm Cov}					% covariance
		\DeclareMathOperator{\SE}{\rm SE}					% standard error
		\DeclareMathOperator{\ssreg}{{\rm SS}_{{\rm Reg}}}	% sum of squared regression
		\DeclareMathOperator{\ssr}{{\rm SS}_{{\rm Res}}}	% sum of squared residuals
		\DeclareMathOperator{\sst}{{\rm SS}_{{\rm Tot}}}	% total sum of squares

	%%%%%%%%%%%%%%%%%%%%%%%%%%%%%%%%%%%%%%%%%%%%%%%%%%%%%%%%%%%%%%%%%%%%%%%%%%%%%%
	%%%%%%%%%%%%%%%%%%%%%%%%%%%%%%% number theory %%%%%%%%%%%%%%%%%%%%%%%%%%%%%%%%
		\renewcommand{\mod}[1]{\ (\mathrm{mod}\ #1)}		% congruences

	%%%%%%%%%%%%%%%%%%%%%%%%%%%%%%%%%%%%%%%%%%%%%%%%%%%%%%%%%%%%%%%%%%%%%%%%%%%%%%
	%%%%%%%%%%%%%%%%%%%%%%%%%%%% theorem environments %%%%%%%%%%%%%%%%%%%%%%%%%%%%
		% Some theorem-like environments, all numbered together starting at 1
		% in each section.

		\newtheorem{thm}{Theorem}[section]					% The default \theoremstyle is 
		\newtheorem{defn}[thm]{Definition}					% bold headings and italic body text.
		\newtheorem{prop}[thm]{Proposition}
		\newtheorem{claim}[thm]{Claim}
		\newtheorem{cor}[thm]{Corollary}
		\newtheorem{lemma}[thm]{Lemma}

		\theoremstyle{definition}  							% Bold headings and Roman body text.
		\newtheorem{example}[thm]{Example}
		\newtheorem{examples}[thm]{Examples}
		\newtheorem{exercise}[thm]{Exercise}
		\newtheorem{note}[thm]{Note}
		\newtheorem{remark}[thm]{Remark}
		\newtheorem{remarks}[thm]{Remarks}
		\newtheorem{discussion}[thm]{Discussion}

		\newcommand{\dfn}{\textbf} 							% Make defined words bold.
		\newcommand{\mdfn}[1]{\dfn{\mathversion{bold}#1}} 	% Even make math symbols bold

	%%%%%%%%%%%%%%%%%%%%%%%%%%%%%%%%%%%%%%%%%%%%%%%%%%%%%%%%%%%%%%%%%%%%%%%%%%%%%%
	%%%%%%%%%%%%%%%%%%%%%%%%%%%%%%% complex numbers %%%%%%%%%%%%%%%%%%%%%%%%%%%%%%
		\DeclareMathOperator{\Arg}{Arg}						% argument of z \in \C
		\DeclareMathOperator{\re}{Re}						% real component
		\DeclareMathOperator{\im}{Im}						% imaginary component

	%%%%%%%%%%%%%%%%%%%%%%%%%%%%%%%%%%%%%%%%%%%%%%%%%%%%%%%%%%%%%%%%%%%%%%%%%%%%%%
	%%%%%%%%%%%%%%%%%%%%%%%%%%%%%%% various symbols %%%%%%%%%%%%%%%%%%%%%%%%%%%%%%
		\newcommand{\iso}{\cong}						% isometric/congruent
		\newcommand{\ra}{\rightarrow}                   % right arrow
		\newcommand{\Ra}{\Rightarrow}                   % right implies
		\newcommand{\lra}{\longrightarrow}              % long right arrow
		\newcommand{\la}{\leftarrow}                    % left arrow
		\newcommand{\La}{\Leftarrow}                    % left implies
		\newcommand{\lla}{\longleftarrow}               % long left arrow
		\newcommand{\eqra}{\llra{\sim}}                 % equivalence/isomorphism
		\newcommand{\blank}{\underbar{\ \ }}          	% An underscore, as in (__)xV
		% \newcommand{\blank}{-}                          % A hyphen, as in (-)xV
		\newcommand{\Id}{Id}                            % The identity functor
		\newcommand{\und}{\underline}
		\newcommand{\del}{\nabla}						% gradient vector

		\raggedbottom	
\begin{document}
	\title{Homework 6  -- Math 397 \\ \today}
	\author{Alex Thies \\ \href{mailto:athies@uoregon.edu}{\lowercase{athies$@$uoregon.edu}}}

	\maketitle

	\section*{Problem 1}
	Use calculus of variations and the Euler-Lagrange equation to show that the shortest distance between two points is given by a curve which is the straight line between the two points.

	\begin{proof}
	Let $Y(x)$ be the family of curves from $x_{1}$ to $x_{2}$, suppose $y(x) \in Y$ such that $$J(y) = \int_{x_{1}}^{x_{2}} F[x,y,\dot{y}] \, dx$$ is minimal.
	Recall that the integrand of $J$ is the functional $F[x,y,\dot{y}] = \sqrt{1 + (\dot{y})^2}$.
	Further, we know that we can use the Euler-Lagrange Equation to optimize $J$ and (hopefully) find that $y(x)$ is linear.
	The Euler-Lagrange Equation for this case is $$\frac{\partial F}{\partial y} - \frac{d}{dx} \frac{\partial F}{\partial \dot{y}} = 0.$$
	We compute some derivatives.
		\begin{align*}
			\frac{\partial F}{\partial y}\left[ \sqrt{1 + (\dot{y})^2} \right] &= 0, \text{ and;} \\
			\frac{\partial F}{\partial \dot{y}} \left[ \sqrt{1 + (\dot{y})^2} \right] &= \frac{\dot{y}}{\sqrt{1 + (\dot{y})^2}}.
		\end{align*}
	It follows that $$\frac{d}{dx}\left[ \frac{\dot{y}}{\sqrt{1 + (\dot{y})^2}} \right] = 0.$$
	This implies that $\dot{y} = a$ for some constant $a$.
	Integrating shows that $y(x) = ax + b$.
	Hence, the curve $y$ that optimizes $J$ is a line $y(x) = ax + b$, as we aimed to show.
	\end{proof}
	% section problem_1 (end)
	\vspace*{2mm}
	\hrule
	\vspace*{2mm}

	\section*{Problem 2}
	A classic problem: What closed curve of given length encloses the maximum area?
	If you have a closed, non-self-intersecting curve $C$ that is traced out by a clockwise moving point in the time interval $0$ to $T$ and the parametric equations for the curve are $x = x(t)$ and $y = y(t)$, then $$\text{area enclosed by $C = \frac{1}{2}\int_{0}^{T} \left[ y(t)\frac{dx}{dt} - x(t)\frac{dy}{dt} \right] dt$.}$$
	We want to find the curve $C$ that maximizes this integral given a fixed perimeter $$\int \, ds = \int_{0}^{T} \sqrt{\left( \frac{dx}{dt} \right)^{2} + \left( \frac{dy}{dt} \right)^{2}} \, dt.$$
	Note that the integral want to optimize is a bit more complicated and we also have a constraint. 
	The function inside is a function of five variables, $t, x, \dot{x}, y$, and $\dot{y}$, where $\dot{x} = dx/dt$ and $\dot{y} = dy/dt$, and we need to find the pair $x(t)$ and $y(t)$ that maximizes area. 
	Through a process similar to what we saw in class, we can derive a pair of Euler-Lagrange equations that lead to the functions: $$\frac{\partial H}{\partial x} - \frac{d}{dt}\frac{\partial H}{\partial \dot{x}} = 0 \ \ \text{and} \ \ \frac{\partial H}{\partial y} - \frac{d}{dt}\frac{\partial H}{\partial \dot{y}} = 0.$$
	The integrand from the area functional is $F = \tfrac{1}{2}(y\dot{x} - x\dot{y})$ and the integrand from the 2 perimeter constraint is $G = (\dot{x}^{2} + \dot{y}^{2})^{1/2}$. 
	Similar to the catenary problem, this leads to an integrand $H[t, x, \dot{x}, y, \dot{y}] = \tfrac{1}{2}(y\dot{x} - x\dot{y})+\lambda(\dot{x}^{2} + \dot{y}^{2})^{1/2}$.

	\begin{enumerate}[\hspace{6mm}(a)]
		\item Do you know the answer to the question before doing any work? 
		What curve will maximize area?

		\item Use the first Euler-Lagrange equation above to obtain the equation $$y - C_{1} = -\frac{\lambda\dot{x}}{\sqrt{\dot{x}^{2} + \dot{y}^{2}}},$$ where $C_{1}$ is a constant.

		\item Use the second Euler-Lagrange equation above to obtain the equation $$x - C_{2} = -\frac{\lambda\dot{y}}{\sqrt{\dot{x}^{2} + \dot{y}^{2}}},$$ where $C_{2}$ is a constant.

		\item Rather than try to solve the differential equations do some algebra and combine the two formulas to obtain an algebraic description of the curve $C$ that maximizes area. 
		What is the curve?

		\item As a last little bit, if $P$ is the length of the curve $C$ (the perimeter), what is the value of $\lambda$?
	\end{enumerate}
	
		\subsection*{Part (a)}
		Do you know the answer to the question before doing any work? 
		What curve will maximize area?

		\begin{proof}[Solution]
		My initial guess is that the closed curve with maximal enclosed area is a circle.
		I don't have a great explanation behind the intuition leading me to this guess.
		The best way to phrase my thinking is that a unit square has area 1, a unit circle has area $\pi$, and I can't think of a shape that could have a higher 'unit-area' than a circle.
		It almost follows from the definition of a circle, in that it is \emph{all} of the points within a radius about a center point; this just seems like the most perimeter-efficient way to collect points in the plane.
		\end{proof}
		% subsection part_a (end)
		\vspace*{2mm}
		\hrule
		\vspace*{2mm}
		\pagebreak

		\subsection*{Part (b)}
		Use the first Euler-Lagrange equation above to obtain the equation $$y - C_{1} = -\frac{\lambda\dot{x}}{\sqrt{\dot{x}^{2} + \dot{y}^{2}}},$$ where $C_{1}$ is a constant.

		\begin{proof}[Solution]
		Let $H[t, x, \dot{x}, y, \dot{y}] = \tfrac{1}{2}(y\dot{x} - x\dot{y})+\lambda(\dot{x}^{2} + \dot{y}^{2})^{1/2}$.
		For the E-L equation we need the following pieces.
			\begin{align*}
				\frac{\partial H}{\partial x}\left[ \frac{1}{2}(y\dot{x} - x\dot{y})+\lambda(\dot{x}^{2} + \dot{y}^{2})^{1/2} \right] &= \frac{1}{2} \frac{\partial H}{\partial x} (y\dot{x} - x\dot{y}) \\
				&\hspace{5mm} + \lambda \frac{\partial H}{\partial x}\sqrt{\dot{x}^{2} + \dot{y}^{2}}, \\
				&= -\frac{\dot{y}}{2} + 0, \\
				&= -\dot{y}/2. \\
				\\
				\frac{\partial H}{\partial \dot{x}}\left[ \frac{1}{2}(y\dot{x} - x\dot{y})+\lambda(\dot{x}^{2} + \dot{y}^{2})^{1/2} \right] &= \frac{1}{2} \frac{\partial H}{\partial \dot{x}} (y\dot{x} - x\dot{y}) \\
				&\hspace{5mm} + \lambda \frac{\partial H}{\partial \dot{x}}\sqrt{\dot{x}^{2} + \dot{y}^{2}}, \\
				&= \frac{y}{2} + \frac{\lambda \, 2\dot{x}}{2 \sqrt{\dot{x}^{2} + \dot{y}^{2}}}, \\
				&= \frac{y}{2} + \frac{\lambda \dot{x}}{\sqrt{\dot{x}^{2} + \dot{y}^{2}}}.
			\end{align*}
		Now we can use the E-L equation with $\dot{x}$,
			\begin{align*}
				- \frac{\dot{y}}{2} - \frac{d}{dt}\left[ \frac{y}{2} + \frac{\lambda \dot{x}}{\sqrt{\dot{x}^{2} + \dot{y}^{2}}} \right] &= 0, \\
				\frac{d}{dt}\left[ -\frac{y}{2} \right] - \frac{d}{dt}\left[ \frac{y}{2} + \frac{\lambda \dot{x}}{\sqrt{\dot{x}^{2} + \dot{y}^{2}}} \right] &= 0, \\
				\frac{d}{dt}\left[ -\frac{y}{2} - \frac{y}{2} - \frac{\lambda \dot{x}}{\sqrt{\dot{x}^{2} + \dot{y}^{2}}} \right] &= 0, \\
				\frac{d}{dt}\left[ -y - \frac{\lambda \dot{x}}{\sqrt{\dot{x}^{2} + \dot{y}^{2}}} \right] &= 0.
			\end{align*}
		Therefore, we have 
			\begin{align*}
				-y - \frac{\lambda \dot{x}}{\sqrt{\dot{x}^{2} + \dot{y}^{2}}} &= C_{1}, \\
				y - C_{1} &= - \frac{\lambda \dot{x}}{\sqrt{\dot{x}^{2} + \dot{y}^{2}}}.
			\end{align*}
		for some constant $C_{1}$, as we aimed to show.			
		\end{proof}
		% subsection part_b (end)
		\vspace*{2mm}
		\hrule
		\vspace*{2mm}

		\subsection*{Part (c)}
		Use the second Euler-Lagrange equation above to obtain the equation $$x - C_{2} = -\frac{\lambda\dot{y}}{\sqrt{\dot{x}^{2} + \dot{y}^{2}}},$$ where $C_{2}$ is a constant.

		\begin{proof}[Solution]
		Let $H[t, x, \dot{x}, y, \dot{y}] = \tfrac{1}{2}(y\dot{x} - x\dot{y})+\lambda(\dot{x}^{2} + \dot{y}^{2})^{1/2}$.
		For the E-L equation we need the following pieces.
			\begin{align*}
				\frac{\partial H}{\partial y}\left[ \frac{1}{2}(y\dot{x} - x\dot{y})+\lambda(\dot{x}^{2} + \dot{y}^{2})^{1/2} \right] &= \frac{1}{2} \frac{\partial H}{\partial y} (y\dot{x} - x\dot{y}) \\
				&\hspace{5mm} + \lambda \frac{\partial H}{\partial y}\sqrt{\dot{x}^{2} + \dot{y}^{2}}, \\
				&= \frac{\dot{x}}{2} + 0, \\
				&= \dot{x}/2. \\
				\\
				\frac{\partial H}{\partial \dot{y}}\left[ \frac{1}{2}(y\dot{x} - x\dot{y})+\lambda(\dot{x}^{2} + \dot{y}^{2})^{1/2} \right] &= \frac{1}{2} \frac{\partial H}{\partial \dot{y}} (y\dot{x} - x\dot{y}) \\
				&\hspace{5mm} + \lambda \frac{\partial H}{\partial \dot{y}}\sqrt{\dot{x}^{2} + \dot{y}^{2}}, \\
				&= -\frac{x}{2} + \frac{\lambda \, 2\dot{y}}{2 \sqrt{\dot{x}^{2} + \dot{y}^{2}}}, \\
				&= -\frac{x}{2} + \frac{\lambda \dot{y}}{\sqrt{\dot{x}^{2} + \dot{y}^{2}}}.
			\end{align*}
		Now we can use the E-L equation with $\dot{y}$,
			\begin{align*}
				\frac{\dot{x}}{2} - \frac{d}{dt}\left[ -\frac{x}{2} + \frac{\lambda \dot{y}}{\sqrt{\dot{x}^{2} + \dot{y}^{2}}} \right] &= 0, \\
				\frac{d}{dt}\left[ \frac{x}{2} \right] - \frac{d}{dt}\left[ -\frac{x}{2} + \frac{\lambda \dot{y}}{\sqrt{\dot{x}^{2} + \dot{y}^{2}}} \right] &= 0, \\
				\frac{d}{dt}\left[ \frac{x}{2} + \frac{x}{2} - \frac{\lambda \dot{y}}{\sqrt{\dot{x}^{2} + \dot{y}^{2}}} \right] &= 0, \\
				\frac{d}{dt}\left[ x - \frac{\lambda \dot{y}}{\sqrt{\dot{x}^{2} + \dot{y}^{2}}} \right] &= 0.
			\end{align*}
		Therefore, we have 
			\begin{align*}
				x - \frac{\lambda \dot{x}}{\sqrt{\dot{x}^{2} + \dot{y}^{2}}} &= C_{2}, \\
				x - C_{2} &= \frac{\lambda \dot{x}}{\sqrt{\dot{x}^{2} + \dot{y}^{2}}}.
			\end{align*}
		for some constant $C_{2}$, as we aimed to show.		
		\end{proof}
		% subsection part_c (end)
		\vspace*{2mm}
		\hrule
		\vspace*{2mm}

		\subsection*{Part (d)}
		Rather than try to solve the differential equations do some algebra and combine the two formulas to obtain an algebraic description of the curve $C$ that maximizes area. 
		What is the curve?

		\begin{proof}[Solution]
		I used algebra a few different ways to solve for $C = C_{2}-C_{1}$, to solve for $dy/dx$ as well as $dx/dy$.
		Finally I solved for $y$ and $x$ just to see if that would show me anything.
		Unfortunately, in each case I was unable to arrive at a result that lended itself to using a de solver in sage.
		\end{proof}
		% subsection part_d (end)
		\vspace*{2mm}
		\hrule
		\vspace*{2mm}

		\subsection*{Part (e)}
		As a last little bit, if $P$ is the length of the curve $C$ (the perimeter), what is the value of $\lambda$?

		\begin{proof}[Solution]
		My guess is that $\lambda = C/2\pi$, but that's just a guess.
		\end{proof}
		% subsection part_e (end)
	% section problem_2 (end)
\end{document}

% http://www.uoduckstore.com/book-search-results?crn=40781,41279,40790,40788,40784 books