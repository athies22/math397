%!TEX output_directory = temp
\documentclass[letterpaper, 12pt]{amsart}
	%%%%%%%%%%%%%%%%%%%%%%%%%%%%%%%%%%%%%%%%%%%%%%%%%%%%%%%%%%%%%%%%%%%%%%%%%%%%%%
	%%%%%%%%%%%%%%%%%%%%%%%%%%%% boilerplate packages %%%%%%%%%%%%%%%%%%%%%%%%%%%%
	\usepackage{amsmath,amssymb,amsthm}
	\usepackage[mathscr]{euscript}
	\usepackage{enumerate}
	\usepackage{graphicx}
	\usepackage{mathrsfs}
	\usepackage{color}
	\usepackage{hyperref}
	\usepackage{verbatim}
	\usepackage{stmaryrd}
	\usepackage[margin=1.25in]{geometry}

	\raggedbottom

	%%%%%%%%%%%%%%%%%%%%%%%%%%%%%%%%%%%%%%%%%%%%%%%%%%%%%%%%%%%%%%%%%%%%%%%%%%%%%%
	%%%%%%%%%%%%%%%%%%%%%%%%%%%%% rename the abstract %%%%%%%%%%%%%%%%%%%%%%%%%%%%
	% \renewcommand{\abstractname}{Introduction}

	%%%%%%%%%%%%%%%%%%%%%%%%%%%%%%%%%%%%%%%%%%%%%%%%%%%%%%%%%%%%%%%%%%%%%%%%%%%%%%
	%%%%%%%%%%%%%%%%%%%%%%%%%%%%%%%%%%%%% sets %%%%%%%%%%%%%%%%%%%%%%%%%%%%%%%%%%%
		%% sets 
		\DeclareMathOperator{\N}{\mathbb{N}}
		\DeclareMathOperator{\Z}{\mathbb{Z}}
		\DeclareMathOperator{\Zp}{\mathbb{Z}^{+}}
		\DeclareMathOperator{\Q}{\mathbb{Q}}
		\DeclareMathOperator{\Qp}{\mathbb{Q}^{+}}
		\DeclareMathOperator{\Qc}{\mathbb{Q}^{c}}
		\DeclareMathOperator{\R}{\mathbb{R}}
		\DeclareMathOperator{\F}{\mathbb{F}}
		\DeclareMathOperator{\Rp}{\mathbb{R}^{+}}
		\DeclareMathOperator{\C}{\mathbb{C}}
		\DeclareMathOperator{\Cnon}{\mathbb{C}^{\times}}
		%% powerset of a set
		\DeclareMathOperator{\pset}{\mathcal{P}}
		%% set of continuous functions in a certain variable
		\DeclareMathOperator{\cont}{\mathscr{C}}
		%% set of functions in a certain variable
		\DeclareMathOperator{\func}{\mathscr{F}}
		%% area function
		\DeclareMathOperator{\A}{\mathcal{A}}
		
	%%%%%%%%%%%%%%%%%%%%%%%%%%%%%%%%%%%%%%%%%%%%%%%%%%%%%%%%%%%%%%%%%%%%%%%%%%%%%%
	%%%%%%%%%%%%%%%%%%%%%%%%%%%%%%%% linear algebra %%%%%%%%%%%%%%%%%%%%%%%%%%%%%%
		%% linear span
		\DeclareMathOperator{\Ell}{\mathscr{L}}
		%% bold vectors with arrows, and bold matrices
		\newcommand{\bmat}[1]{{\mathbf{#1}}}
		\newcommand{\bvec}[1]{{\vec{\mathbf{#1}}}}
		%% independent vectors/matrices
		\DeclareMathOperator{\ind}{\perp\!\!\!\perp}
		%% order
		\DeclareMathOperator{\ord}{\text{ord}}

	%%%%%%%%%%%%%%%%%%%%%%%%%%%%%%%%%%%%%%%%%%%%%%%%%%%%%%%%%%%%%%%%%%%%%%%%%%%%%%
	%%%%%%%%%%%%%%%%%%%%%%%%%%% probability & statistics %%%%%%%%%%%%%%%%%%%%%%%%%
		%% probability, expectation, variance, etc.
		\renewcommand{\Pr}{\mathbb{P}}
		\DeclareMathOperator{\E}{\mathbb{E}}
		\DeclareMathOperator{\var}{\rm Var}
		\DeclareMathOperator{\sd}{\rm SD}
		\DeclareMathOperator{\cov}{\rm Cov}
		\DeclareMathOperator{\SE}{\rm SE}
		\DeclareMathOperator{\ssreg}{{\rm SS}_{{\rm Reg}}}
		\DeclareMathOperator{\ssr}{{\rm SS}_{{\rm Res}}}
		\DeclareMathOperator{\sst}{{\rm SS}_{{\rm Tot}}}

	%%%%%%%%%%%%%%%%%%%%%%%%%%%%%%%%%%%%%%%%%%%%%%%%%%%%%%%%%%%%%%%%%%%%%%%%%%%%%%
	%%%%%%%%%%%%%%%%%%%%%%%%%%%%%%%% congruences %%%%%%%%%%%%%%%%%%%%%%%%%%%%%%%%%
		\renewcommand{\mod}[1]{\ (\mathrm{mod}\ #1)}

	%%%%%%%%%%%%%%%%%%%%%%%%%%%%%%%%%%%%%%%%%%%%%%%%%%%%%%%%%%%%%%%%%%%%%%%%%%%%%%
	%%%%%%%%%%%%%%%%%%%%%%%%%%%%%% bracket notation %%%%%%%%%%%%%%%%%%%%%%%%%%%%%%
		% I first used this for principal ideals, that is why the abbreviation is pid
		\newcommand{\pid}[1]{\langle #1 \rangle}

	%%%%%%%%%%%%%%%%%%%%%%%%%%%%%%%%%%%%%%%%%%%%%%%%%%%%%%%%%%%%%%%%%%%%%%%%%%%%%%
	%%%%%%%%%%%%%%%%%%%%%%%%%%%%%%% fatdot notation %%%%%%%%%%%%%%%%%%%%%%%%%%%%%%
		\makeatletter
			\newcommand*\fatdot{\mathpalette\fatdot@{.5}}
			\newcommand*\fatdot@[2]{\mathbin{\vcenter{\hbox{\scalebox{#2}{$\m@th#1\bullet$}}}}}
		\makeatother

	%%%%%%%%%%%%%%%%%%%%%%%%%%%%%%%%%%%%%%%%%%%%%%%%%%%%%%%%%%%%%%%%%%%%%%%%%%%%%%
	%%%%%%%%%%%%%%%%%%%%%%%%%%%%%% use pretty letters %%%%%%%%%%%%%%%%%%%%%%%%%%%%
		\DeclareMathOperator{\ep}{\varepsilon}
		\DeclareMathOperator{\ph}{\varphi}

	%%%%%%%%%%%%%%%%%%%%%%%%%%%%%%%%%%%%%%%%%%%%%%%%%%%%%%%%%%%%%%%%%%%%%%%%%%%%%%
	%%%%%%%%%%%%%%%%%%%%%%%%%%% stolen from Jeske/Dugger %%%%%%%%%%%%%%%%%%%%%%%%%
	% Some theorem-like environments, all numbered together starting at 1
	% in each section.

	% The default \theoremstyle is bold headings and italic body text.
	\newtheorem{thm}{Theorem}[section]
	\newtheorem{axe}[thm]{Axiom}
	\newtheorem{cn}[thm]{Common Notion}
	\newtheorem{defn}[thm]{Definition}
	\newtheorem{prop}[thm]{Proposition}
	\newtheorem{claim}[thm]{Claim}
	\newtheorem{cor}[thm]{Corollary}
	\newtheorem{lemma}[thm]{Lemma}

	\theoremstyle{definition}  % Bold headings and Roman body text.
	\newtheorem{example}[thm]{Example}
	\newtheorem{examples}[thm]{Examples}
	\newtheorem{exercise}[thm]{Exercise}
	\newtheorem{note}[thm]{Note}
	\newtheorem{remark}[thm]{Remark}
	\newtheorem{remarks}[thm]{Remarks}
	\newtheorem{discussion}[thm]{Discussion}

	\newcommand{\dfn}{\textbf} % Make defined words bold.
	\newcommand{\mdfn}[1]{\dfn{\mathversion{bold}#1}} % Even make math symbols bold

	% Various commands that are useful.  Please add your own.

	\DeclareMathOperator{\Arg}{Arg}
	\DeclareMathOperator{\re}{Re}
	\DeclareMathOperator{\im}{Im}
	\DeclareMathOperator{\Log}{Log}
	\DeclareMathOperator{\Span}{Span}

	\newcommand{\iso}{\cong}						% isometric/congruent
	\newcommand{\ra}{\rightarrow}                   % right arrow
	\newcommand{\Ra}{\Rightarrow}                   % right implies
	\newcommand{\lra}{\longrightarrow}              % long right arrow
	\newcommand{\la}{\leftarrow}                    % left arrow
	\newcommand{\La}{\Leftarrow}                    % left implies
	\newcommand{\lla}{\longleftarrow}               % long left arrow
	\newcommand{\llra}[1]{\stackrel{#1}{\lra}}      % labeled long right arrow
	\newcommand{\we}{\llra{\sim}}                   % weak equivalence
	\newcommand{\cof}{\rightarrowtail}              % cofibration
	\newcommand{\fib}{\twoheadrightarrow}           % fibration
	\newcommand{\inc}{\hookrightarrow}              % inclusion
	\newcommand{\dbra}{\rightrightarrows}           % double arrow for equalizer diagrams
	\newcommand{\eqra}{\llra{\sim}}                 % equivalence/isomorphism

	% \newcommand{\blank}{\underbar{\ \ }}          % An underscore, as in (__)xV
	\newcommand{\blank}{-}                          % A hyphen, as in (-)xV
	\newcommand{\Id}{Id}                            % The identity functor
	\newcommand{\und}{\underline}
	\newcommand{\norm}[1]{\mid \!\!#1 \!\!\mid}             %\norm{x} gives |x|

	% These commands are for the period and comma in the lower right entry of
	% a diagram.  They put the punctuation 2 pts to the right, but make
	% TeX (and hence the diagram package) unaware of the extra width
	% of that entry.
	\newcommand{\period}    {{\makebox[0pt][l]{\hspace{2pt} .}}}
	\newcommand{\comma}     {{\makebox[0pt][l]{\hspace{2pt} ,}}}
	\newcommand{\semicolon} {{\makebox[0pt][l]{\hspace{2pt} ;}}}

	\newcommand{\Cech}{\v{C}ech}
	\newcommand{\scat}{\Delta}
	\newcommand{\assign}{\ra}
	\newcommand{\copr}{\,\amalg\,}
	\newcommand{\ovcat}{\downarrow}
	\newcommand{\pder}[2]{{\frac{\partial #1}{\partial #2}}}
	\newcommand{\del}{\nabla}
	\newcommand{\vectr}[1]{{\mbox{\boldmath $#1$}}}
	\newcommand{\uvectr}[1]{\vectr{\hat #1}}
	\newcommand{\ihat}{\uvectr \imath}
	\newcommand{\jhat}{\uvectr \jmath}
	\newcommand{\khat}{\uvectr k}
	\newcommand{\rhat}{\uvectr r}
	\newcommand{\thhat}{\uvectr \theta}
	\newcommand{\zhat}{\uvectr z}
	\newcommand{\rhohat}{\uvectr \rho}
	\newcommand{\phihat}{\uvectr \phi}
	\newcommand{\grad}{\vectr{\vec\nabla}}
	% \newcommand{\R}{\mathbb{R}}
	\newcommand{\vv}[1]{\vectr{v_{#1}}}
	\newcommand{\crad}{0.1}
	\newcommand{\lline}[1]{\overleftrightarrow{#1}}
	\DeclareMathOperator{\area}{area}
	\DeclareMathOperator{\vol}{vol}
	\newcommand{\ray}[1]{\overset{\rightarrow}{#1}}
	\newcommand{\sr}[2]{???}
	\newcommand{\iihat}{i}
	\newcommand{\jjhat}{j}
	\newcommand{\kkhat}{k}

	\renewcommand{\abstractname}{Comment}	

	\usepackage{pgf,tikz}
	\usetikzlibrary{arrows}	
\begin{document}
	\title{On the Cusp of Calculus \\ \today}
	\author{Alex Thies \\ \href{mailto:athies@uoregon.edu}{\lowercase{athies$@$uoregon.edu}}}

	\pagenumbering{gobble}
	\maketitle
	\newpage

	\pagenumbering{roman}
	\setcounter{tocdepth}{1}
	\tableofcontents
	\newpage

	\pagenumbering{arabic}
	\section{Introduction}
	\label{sec:introduction}
	More than two millenia ago, the Ancient Greeks were becoming the first civilization to bring logical rigor to their mathematics.
	Even more impressively, they were taking care to record their mathematics in tomes that are still taught in classrooms today.
	As a result of their logical, axiomatic structures, their mathematical researches led to results that challenged philosophical and religious dogmas.
	To their credit, the mathematicians of Ancient Greece changed their views according to the uncomfortable truths they uncovered through mathematics.\footnote{For example, there was a time when Ancient Greek metaphysics assumed that any measurable thing, i.e., anything with a magnitude, could be composed of incommensurable units. This manifested in the atom hypothesis, centuries ahead of of its time. It also required that irrational numbers such as $\sqrt{2}$ could not exist, when they in fact do. This was discovered by a student of Pythagoras when examining a unit square, whose diagonal is of length $\sqrt{2}$. Although this upset the Pythagoreans at the time, centuries later Euclid would go on to include in \textit{The Elements} an entire book on irrational numbers (Book X, the longest book in \textit{The Elements}).}
	Among the greatest mathematicians of Ancient Greece are the familiar names of Euclid, Pythagoras, and Eudoxus.
	Rising above them all was Archimedes of Syracuse.

	If you were to ask a layman what Archimedes was best known for, if you got any answer, it would likely be that he used the varying densities of metals and their displacement in water to show that a crown that was supposedly made of pure gold was in fact not pure gold.
	That on its own was a particularly clever discovery, but it is dwarfed by Archimedes' greatest achievements.
	Furthermore, this example underscores that beyond mathematics, Archimedes made notable contributions to physics and engineering.
	He invented the Archimedes Screw, a device helps farmers with irrigation; after that, he developed the engineering fundamentals for the lever, and used them to great effect in wartime.\textbf{[5]}
	In the field of mathematics, without other fields of mathematics in which to play, Archimedes pushed the bounds of Euclidean Geometry to heights that would unmatched for centuries.
	Chief among his mathematical accomplishments, Archimedes developed what he titled the \textit{Method of Exhaustion}, a tool that (among other things) enables one create an algorithm that could achieve an approximation of $\pi$ to any desired level of accuracy.\textbf{[3]}
	A cousin of this algorithm allowed him to compute the area of a parabolic section.\textbf{[2]}
	
	Were it not for the contemporary philisophical concerns about the concepts of the infinitely-small and infinity-large, Archimedes might have been able to broach the subject of analysis by creating the precursor to the limit.\textbf{[3]}
	Analysis, specifically Calculus, is grounded in the study of limits.
	This was not always true in the modern, rigorous sense, in that the derivative is defined as the limit of a difference quotient, and the integral is defined as the limit of a riemann sum.
	Nonetheless, Calculus has always been the study of infinitely-small rates of change and their relationship to instantaneous rates of change, as well as the study of using infinitely-many inscribed figures to approximate the area of a lager figure.
	Archimedes came amazingly close to the latter of these branches of Calculus with his \textit{Method of Exhaustion}.
	It was by this method that he achieved the mathematical advances listed above.

	Students of Calculus will be familiar with the concept of approximating the area of a figure by using infinitely-many rectangles to fill the figure, this is a nonmathematicians's description of the integral.
	Instead of the radioactive concepts of infinitely-many rectangles, Archimedes' clever mind created an algorithm that allows for the creation of as many polygons as one wanted.
	It was an algorithm that could have infinite run-time, creating better approximations by each iteration.

	We will show \textit{The Method of Exhaustion}, as well as two of its applications.
	Going further, we will describe a few of the mathematical descendants of \textit{The Method}, underscoring its genius and that of its inventor.
	% section introduction (end)

	\section{The Method of Exhaustion}
	\label{sec:the_method_of_exhaustion}
		\subsection{Axioms and Definitions}
		\label{sub:axioms_and_definitions}
		These are modern restatements of the axioms that Archimedes used in the crafting of his \textit{Method}.
		Let $\Omega_{2}$ denote the set of all two-dimension shapes in the plane.

		\begin{axe}
		Let $S,T \in \Omega_{2}$, if $S \subset T$, then $\A(S) \leq \A(T)$.
		\end{axe}

		\begin{axe}
		Let $S,T \in \Omega_{2}$ such that $S \cap T = \emptyset$, consider $R = S \cup T$, then $\A(R) = \A(S) + \A(T)$.
		\end{axe}

		Additionally, Archimedes' Method of Exhaustion required a principle that we now attribute to the Greek mathematician Eudoxus; what follows is a modern statement of what can be found as Proposition 1 from Book X of \textit{The Elements}.
		\begin{axe}[Eudoxus' Principle]
		\label{axe:eudoxus}
		Given two magnitudes $a$ and $b$, there exists $n \in \N$ such that $na>b$.
		\end{axe}		
		% subsection axioms_and_definitions (end)

		\subsection{The Method}
		\label{sub:the_method}
		Our goal with \textit{The Method} will be, given any magnitude $M$, and small real number $\ep > 0$, to construct new, smaller, magnitudes from $M$, such that one of these new magnitudes will be less than $\ep$.
		In other words, \textit{The Method} is what allows us to make something as small as we'd wish.

		Let $M_{0}, \ep \in \R^{+}$, and $M_{1}, M_{2}, \dots$ such that $M_{1} < \tfrac{1}{2}M_{0}, \, M_{2} < \tfrac{1}{2}M_{1}, \dots$.
		Then, we want to continue making new $M_{i}$'s, until we get $M_{N} < \ep$ for some $N \in \N$.
		By Eudoxus' Principle, choose $N$ such that $(N+1)\ep > M_{0}$.
		Since $N \in \N$, we know $N+1 \geq 2$, thus $\ep \leq \frac{1}{2}(N+1)\ep$.
		This gives us $(N+1)\ep - \ep > \tfrac{1}{2}(N+1)\ep > \tfrac{1}{2}M_{0}$.
		Therefore $N\ep > \tfrac{1}{2}M_{0} > M_{1}$, which implies $N\ep > M_{1}$.

		If we repeat this process, we get $(N-1)\ep > M_{2}$, until eventually we arrive at $(N - (N-1))\ep = \ep > M_{N}$, thereby achieving our desired goal of constructing a magnitude less than $\ep$.
		% subsection the_method (end)
	% section the_method_of_exhaustion (end)

	\section{Applications}
	\label{sec:applications}
		\subsection{Approximation of $\pi$}
		\label{sub:approximation_of_pi}
		We begin with some definitions:
		\begin{defn}
		\label{defn:fig}
		For the sake of mathematical rigor, we define a two-dimensional shape in the plane as a \textbf{figure}; the sides of the figure, or the extremities, are called the \textbf{boundary} of the figure; if the boundary of a figure is composed of only straight lines, then we say that figure is \textbf{rectilinear}, or \textbf{polygonal}; if the boundary of a figure contains at least one segment that is not a straight line, then we say that figure is \textbf{curvilinear}.
		\end{defn}

		\begin{defn}
		\label{defn:circle}
		We define a \textbf{circle} to be the set $C = \{ (x,y) : (x - h)^{2} + (y - k)^{2} = r^{2} \}$ of all points that are distance $r$ from a point $P = (h,k)$; we say that $r$ is the \textbf{radius}, and that $P$ is the \textbf{center}.
		\end{defn}

		\begin{defn}
		\label{defn:polygon}
		Let $P_{1}P_{2} \cdots P_{n}$ be a set of points in the plane that can be connected by non-intersecting straight-line segments, these non-intersecting straight-line segments connected by $P_{1}P_{2} \cdots P_{n}$ form the boundary of a figure called $P_{1}P_{2} \cdots P_{n}$.
		We say that $P_{1}P_{2} \cdots P_{n}$ is a \textbf{polygon}, that $P_{1}P_{2} \cdots P_{n}$ are its \textbf{vertices}, that $\overline{P_{i}P_{i+1}}$ are its \textbf{sides}, and that $P_{i}P_{i+1} = a_{i}$ are the \textbf{lengths}, or \textbf{magnitudes} of its sides.

		If a polygon $P$ is such that its sides are all of the same length, and its interior angles are all of the same measure, then and only then is $P$ called a \textbf{regular polygon}; if $P$ has $n$ sides then it can be called a \textbf{regular n-gon}.
		\end{defn}

		\begin{defn}
		\label{defn:sim_polygons}
		We say that two polygonal figures $P$ and $Q$ in the plane are \textbf{similar} if and only if the ratio of sides of $P$ to their corresponding sides in $Q$ are proportional.
		\end{defn}

		\begin{defn}
		\label{defn:area_function}
		Let $\, \Omega_{2}$ be the set of all two-dimensional shapes in the plane, and let $P \in \Omega_{2}$.
		We define $\A : \Omega_{2} \to \R^{+}$, a function that takes a shape from the plane and outputs the area of the shape, i.e., $\A(P) \mapsto x$ where $x \in \R$ is the area of $P$.
		We call this function the \textbf{area function}, it is defined piece-wise, but given the myriad of irregular shapes in a two-dimensional plane, it is far too cumbersome to explicitly define each mapping that $\A$ can take.
		One mapping that the reader will be familiar with is for a square $ABCD$ with side length $AB = s$, we have $\A(ABCD) = s^{2}$, another is for $\triangle ABC$ with base $b$ and height $h$ we have $\A(\triangle ABC) = (1/2)bh$.
		\end{defn}
		Now that we've agreed on what to call things, let's get into the problem at hand.

		The application of \textit{The Method of Exhaustion} with which Archimedes has gained the most praise is his approximation of $\pi$.
		His was not the first approximation of $\pi$, there is evidence in the historical record that the Babylonians, Sumerians, and Egyptians each had approximations at least as accurate as $22/7 \doteq 3.14$, the approximation used in middle school classrooms across the United States today.
		The major upshot of the \textit{Method} is that it can be iterated in order to achieve any level of accuracy one has the time to compute by painstaking geometric construction.

		Archimedes' idea was to inscribe a given circle $C$ with a regular $n$-gon $P$ and take the area of $P$ as a lower-bound for approximating the area of $C$.
		In a similar fashion, we circumscribe $C$ with a regular $n$-gon $Q$ and take the area of $Q$ as an upper-bound for approximating the area of $C$.
		Thus, we have $$\A(P) < \A(C) < \A(Q).$$
		Next is when Archimedes was particularly clever.
		Let $P_{1}$ be a regular $2n$-gon inscribed in $C$, and let $Q_{1}$ be a regular $2n$-gon circumscribed about $C$.
		The areas of these new inscribed and circumscribing polygons provide more accurate upper- and lower-bound estimates for approximating $\A(C)$, thus $$\A(P) < \A(P_{1}) < \A(C) < \A(Q_{1}) < \A(Q).$$
		This enables the use of the \textit{Method of Exhaustion} to construct regular $n$-gons with ever increasing values for $n$, and ever decreasing error in the approximation of the area of $C$.

		This brings us to our first lemma, which allows us to inscribe into a circle, regular $n$-gons with ever increasing number of sides, so that we can approximate the area of a circle by computing the area of our inscribed regular $n$-gons, and continue to making $n$ larger until we can make the error between the area of the $n$-gon and the circle to be as small as we need.

		\begin{lemma}
		Given a circle $C$ and a small number $\ep \in \R$ with $\ep > 0$, there exists a regular polygon $P$ inscribed in $C$ such that $\A(C) - \A(P) < \ep$.
		\end{lemma}
		\begin{proof}
		Let $C$ be a circle in the plane with center $O$ and radius $r$; let $P_{0} = ABCD$ be a square which is inscribed in $C$.

		\begin{figure}[b]
			\definecolor{yqyqyq}{rgb}{0.5019607843137255,0.5019607843137255,0.5019607843137255}
			\definecolor{qqqqff}{rgb}{0.,0.,1.}
			\definecolor{ffqqff}{rgb}{1.,0.,1.}
			\definecolor{uuuuuu}{rgb}{0.26666666666666666,0.26666666666666666,0.26666666666666666}
			\definecolor{qqffqq}{rgb}{0.,1.,0.}
			\definecolor{xdxdff}{rgb}{0.49019607843137253,0.49019607843137253,1.}
			\definecolor{ududff}{rgb}{0.30196078431372547,0.30196078431372547,1.}

			\resizebox {0.65\textwidth} {!} {
			\begin{tikzpicture}[line cap=round,line join=round,>=triangle 45,x=1.0cm,y=1.0cm]
			\clip(-2.2057323087868768,-3.4036655878360578) rectangle (8.918616299547809,7.208826489440268);
			\draw[line width=2.pt,color=yqyqyq,fill=yqyqyq,fill opacity=0.10000000149011612] (5.394938336631308,3.4297144668130515) -- (5.654106899755302,3.6823608799714127) -- (5.401460486596941,3.9415294430954066) -- (5.142291923472947,3.6888830299370454) -- cycle; 
			\draw[line width=2.pt,color=yqyqyq,fill=yqyqyq,fill opacity=0.10000000149011612] (5.7687276083319645,4.805011003024506) -- (5.50955904520797,4.552364589866145) -- (5.7622054583663305,4.293196026742152) -- (6.021374021490325,4.545842439900512) -- cycle; 
			\draw [line width=2.pt] (3.02,1.62) circle (4.191515239146817cm);
			\draw [line width=2.pt,color=qqffqq] (3.073408893535901,5.811174953409995)-- (7.211174953409994,1.566591106464099);
			\draw [line width=2.pt,color=qqffqq] (7.211174953409994,1.566591106464099)-- (2.966591106464099,-2.5711749534099946);
			\draw [line width=2.pt,color=qqffqq] (2.966591106464099,-2.5711749534099946)-- (-1.1711749534099942,1.6734088935359013);
			\draw [line width=2.pt,color=qqffqq] (-1.1711749534099942,1.6734088935359013)-- (3.073408893535901,5.811174953409995);
			\draw [line width=2.pt,color=ffqqff] (3.073408893535901,5.811174953409995)-- (6.021374021490325,4.545842439900512);
			\draw [line width=2.pt,color=ffqqff] (6.021374021490325,4.545842439900512)-- (7.211174953409994,1.566591106464099);
			\draw [line width=2.pt,color=ffqqff] (7.211174953409994,1.566591106464099)-- (5.945842439900513,-1.3813740214903254);
			\draw [line width=2.pt,color=ffqqff] (5.945842439900513,-1.3813740214903254)-- (2.966591106464099,-2.5711749534099946);
			\draw [line width=2.pt,color=ffqqff] (2.966591106464099,-2.5711749534099946)-- (0.018625978509674904,-1.3058424399005133);
			\draw [line width=2.pt,color=ffqqff] (0.018625978509674904,-1.3058424399005133)-- (-1.1711749534099942,1.6734088935359013);
			\draw [line width=2.pt,color=ffqqff] (-1.1711749534099942,1.6734088935359013)-- (0.09415756009948772,4.621374021490325);
			\draw [line width=2.pt,color=ffqqff] (0.09415756009948772,4.621374021490325)-- (3.073408893535901,5.811174953409995);
			\draw [line width=2.pt,dash pattern=on 4pt off 4pt,color=qqqqff] (3.02,1.62)-- (3.073408893535901,5.811174953409995);
			\draw [line width=2.pt,dash pattern=on 4pt off 4pt,color=qqqqff] (3.02,1.62)-- (7.211174953409994,1.566591106464099);
			\draw [line width=2.pt,dash pattern=on 4pt off 4pt,color=qqqqff] (3.02,1.62)-- (6.021374021490325,4.545842439900512);
			\draw [line width=2.pt,dotted,color=qqffqq] (3.073408893535901,5.811174953409995)-- (3.9524909915532804,6.668134363373463);
			\draw [line width=2.pt,dotted,color=qqffqq] (3.9524909915532804,6.668134363373463)-- (8.090257051427374,2.423550516427567);
			\draw [line width=2.pt,dotted,color=qqffqq] (8.090257051427374,2.423550516427567)-- (7.211174953409994,1.566591106464099);
			\begin{scriptsize}
			\draw [fill=ududff] (3.02,1.62) circle (2.5pt);
			\draw[color=ududff] (2.6398428518987838,1.4504405150339017) node {$O$};
			\draw [fill=xdxdff] (3.073408893535901,5.811174953409995) circle (2.5pt);
			\draw[color=xdxdff] (2.998142423639625,6.244830022613721) node {$A$};
			\draw [fill=xdxdff] (2.966591106464099,-2.5711749534099946) circle (2.5pt);
			\draw[color=xdxdff] (3.0322661923768477,-2.8832781145934088) node {$C$};
			\draw [fill=xdxdff] (7.211174953409994,1.566591106464099) circle (2.5pt);
			\draw[color=xdxdff] (7.502479896953056,1.5528118212455704) node {$B$};
			\draw [fill=xdxdff] (-1.1711749534099942,1.6734088935359013) circle (2.5pt);
			\draw[color=xdxdff] (-1.745061430834367,1.7404925493002963) node {$D$};
			\draw [fill=uuuuuu] (6.021374021490325,4.545842439900512) circle (2.0pt);
			\draw[color=uuuuuu] (6.222838569307195,4.931064926230639) node {$E$};
			\draw [fill=uuuuuu] (5.945842439900513,-1.3813740214903254) circle (2.0pt);
			\draw[color=uuuuuu] (6.205776684938583,-1.450079827630046) node {$F$};
			\draw [fill=uuuuuu] (0.018625978509674904,-1.3058424399005133) circle (2.0pt);
			\draw[color=uuuuuu] (-0.41423445008267146,-1.3988941745242116) node {$G$};
			\draw [fill=uuuuuu] (0.09415756009948772,4.621374021490325) circle (2.0pt);
			\draw[color=uuuuuu] (-0.20949183765933366,4.82869362001897) node {$H$};
			\draw[color=qqqqff] (4.977321010398557,1.31394544008501) node {$r$};
			\draw [fill=uuuuuu] (3.9524909915532804,6.668134363373463) circle (2.0pt);
			\draw[color=uuuuuu] (4.124226791967983,6.944367281726791) node {$A'$};
			\draw [fill=uuuuuu] (8.090257051427374,2.423550516427567) circle (2.0pt);
			\draw[color=uuuuuu] (8.253202809171961,2.713019958311149) node {$B'$};
			\end{scriptsize}
			\end{tikzpicture}
			}
			\caption{Construction for $n=0$}
			\label{nzero}
		\end{figure}

		Let $M_{0} = \A(C) - \A(P_{0})$.
		Next, we double the number of sides of $P_{0}$, and create the regular octogon $P_{1}$.
		Continue this process, generating a sequence of regular polygons $P_{0}, P_{1}, \dots, P_{n}$ each with $2^{n+2}$-many sides.
		Let $M_{n} = \A(C) - \A(P_{n})$.
		We want to show that $M_{n} - M_{n+1} > \tfrac{1}{2}M_{n}$, this is because 
			\begin{align*}
			M_{n} - M_{n+1} &> \tfrac{1}{2}M_{n}, \\
			M_{n} - \tfrac{1}{2}M_{n} &> M_{n+1}, \\
			\tfrac{1}{2}M_{n} &> M_{n+1}.
			\end{align*}
		Notice that $\tfrac{1}{2}M_{n} > M_{n+1}$ is precisely the condition that we need to create in order to make use of the Method of Exhaustion.

		Consider $n = 0$:
			\begin{align*}
			M_{0} - M_{1} &= [\A(C) - \A(P_{0})] - [\A(C) - \A(P_{1})], \\
			&= \A(P_{1}) - \A(P_{0}), \\
			&= 4 \A(\triangle ABE), \\
			&= 2 \A(ABB'A'), \\
			&> 2 \A(\frown{ABE}), \\
			&> \tfrac{1}{2} \cdot \left[ 4 \cdot \A(\frown{ABE}) \right], \\
			&> \tfrac{1}{2} \left[ \A(C) - \A(P_{0}) \right], \\
			&> \tfrac{1}{2} M_{0}.
			\end{align*}
		Hence, the claim is true for $n=0$, and it should be clear that if we continue in this manner, changing what needs to be changed along the way, we will arive at the result $M_{n} - M_{n+1} > \tfrac{1}{2}M_{n}$.
		Thus, by \ref{axe:eudoxus} there exists some $N \in \N$ such that $M_{N} < \ep$, as we aimed to show.
		\end{proof}

		Now, we can play the inscribing/circumscribing game as many times as we'd like to achieve bounds for the area of $C$.
		Archimedes' choice was the start with an inscribed hexagon, and to continue to double the number of sides, building several regular $6*(2^{n})$-gons.
		The reason for this strategy is due to the relative ease by which one can construct an inscribed regular 12-gon, given a circle that is already inscribed by a regular hexagon.
		Nonetheless, it remains to show these areas relate to $\pi$.

		The idea that the area and circumference of any circle are related to a universal constant close to $3.14$ was not new to Archimedes; indeed it was known by many of his predecessors.
		With the next two results, we show that indeed the ratio of the area of any circle to the square of its radius -- $\A(C)/r^{2}$ -- is equal to some constant, that we have come to denote $\pi$.
		However, we do not show how this constant relates to the circumference of a circle.
		For those who are interested, \textbf{[3]} and \textbf{[5]} offer good explanations of the interplay between circumference and area.

		\begin{lemma}
		The ratio of areas of two similar regular polygons is proportional to the ratio of the squares of their corresponding sides.
		\end{lemma}
		\begin{proof}
		Let $P$ be a regular $n$-gon with sides $s_{i}$, vertices $P_{i}$ for $1 \leq i \leq n$, and center $P_{0}$.
		Let $T_{i} = \triangle P_{i}P_{i+1}P_{0}$, for $1 \leq i \leq n - 1$ and $T_{n} = \triangle P_{n}P_{1}P_{0}$.
		We'll refer to $T_{i}$ as an \textbf{interior triangle} of $P$.
		The bases of $T_{i}$ are the sides of $P$, i.e. $P_{i}P_{i+1} = s_{i}$, let the heights of these triangles be $h_{i}$, and the radius of $P$ is $P_{0}P_{i} = r$
		Recall $\A(T_{i}) = \tfrac{1}{2}s_{i}h_{i}$.
		Then by Axiom 2, we have that $\A(T_{1}) + \cdots + \A(T_{n}) = \A(P)$, it follows that $\A(P) = n\left( \tfrac{1}{2}s_{i}h_{i} \right)$.
		Let $Q$ be a regular $n$-gon with sides $\sigma_{i} \neq s_{i}$, with interior triangles $V_{i}$ defined as $T_{i}$ were defined for $P$, where the heights of $V_{i}$ is $\eta_{i}$, and radius $Q_{0}Q_{i} = \rho$.
		As with $P$, we can say $\A(Q) = n(\tfrac{1}{2} \sigma_{i}\eta_{i})$.

		\begin{figure}[b]
			\definecolor{yqyqyq}{rgb}{0.5019607843137255,0.5019607843137255,0.5019607843137255}
			\definecolor{qqffff}{rgb}{0.,1.,1.}
			\definecolor{xfqqff}{rgb}{0.4980392156862745,0.,1.}
			\definecolor{uuuuuu}{rgb}{0.26666666666666666,0.26666666666666666,0.26666666666666666}
			\definecolor{qqwuqq}{rgb}{0.,0.39215686274509803,0.}
			\definecolor{ffxfqq}{rgb}{1.,0.4980392156862745,0.}
			\definecolor{xdxdff}{rgb}{0.49019607843137253,0.49019607843137253,1.}
			\definecolor{ududff}{rgb}{0.30196078431372547,0.30196078431372547,1.}

			\resizebox {0.65\textwidth} {!} {
			\begin{tikzpicture}[line cap=round,line join=round,>=triangle 45,x=1.0cm,y=1.0cm]
			\clip(2.66243032205374,1.5070220034545028) rectangle (9.667140883459258,6.849197212265431);
			\draw[line width=2.pt,color=yqyqyq,fill=yqyqyq,fill opacity=0.10000000149011612] (6.03667195629688,5.904804124441411) -- (6.195220623918996,5.851889680207063) -- (6.248135068153346,6.01043834782918) -- (6.089586400531228,6.063352792063529) -- cycle; 
			\draw[line width=2.pt,color=yqyqyq,fill=yqyqyq,fill opacity=0.10000000149011612] (7.144047442813612,5.076532128959121) -- (7.278484723789221,4.977213269233168) -- (7.377803583515174,5.111650550208777) -- (7.243366302539565,5.21096940993473) -- cycle; 
			\draw [line width=2.pt] (4.62,1.66) circle (4.642111588490738cm);
			\draw [line width=2.pt,dash pattern=on 2pt off 2pt,color=ffxfqq] (4.62,1.66)-- (6.089586400531228,6.063352792063529);
			\draw [line width=2.pt,dash pattern=on 2pt off 2pt,color=ffxfqq] (4.62,1.66)-- (8.3971462045479,4.358586027805933);
			\draw [line width=2.pt,dash pattern=on 2pt off 2pt,color=ffxfqq] (4.62,1.66)-- (3.220694540916245,6.086188454209604);
			\draw [line width=2.pt,dash pattern=on 2pt off 2pt,color=qqwuqq] (6.089586400531228,6.063352792063529)-- (8.3971462045479,4.358586027805933);
			\draw [line width=2.pt,dash pattern=on 2pt off 2pt,color=qqwuqq] (3.220694540916245,6.086188454209604)-- (6.089586400531228,6.063352792063529);
			\draw [line width=2.pt,dash pattern=on 2pt off 2pt,color=qqwuqq] (3.220694540916245,6.086188454209604)-- (0.8862898060279809,4.418370567459968);
			\draw [line width=2.pt,dash pattern=on 2pt off 2pt,color=qqwuqq] (8.3971462045479,4.358586027805933)-- (9.261964538904936,1.6230511224920405);
			\draw [line width=2.pt,dash pattern=on 2pt off 2pt,color=qqwuqq] (9.261964538904936,1.6230511224920405)-- (8.35371019397202,-1.0983705674599686);
			\draw [line width=2.pt,dash pattern=on 2pt off 2pt,color=qqwuqq] (8.35371019397202,-1.0983705674599686)-- (6.019305459083755,-2.766188454209604);
			\draw [line width=2.pt,dash pattern=on 2pt off 2pt,color=qqwuqq] (6.019305459083755,-2.766188454209604)-- (3.150413599468772,-2.7433527920635283);
			\draw [line width=2.pt,dash pattern=on 2pt off 2pt,color=qqwuqq] (3.150413599468772,-2.7433527920635283)-- (0.8428537954521014,-1.0385860278059333);
			\draw [line width=2.pt,dash pattern=on 2pt off 2pt,color=qqwuqq] (0.8428537954521014,-1.0385860278059333)-- (-0.02196453890493455,1.6969488775079593);
			\draw [line width=2.pt,dash pattern=on 2pt off 2pt,color=qqwuqq] (-0.02196453890493455,1.6969488775079593)-- (0.8862898060279809,4.418370567459968);
			\draw [line width=2.pt,dash pattern=on 1pt off 1pt,color=xfqqff] (4.6588503489276265,6.540850358912145)-- (7.52032245213483,5.585855225214912);
			\draw [line width=2.pt,dash pattern=on 1pt off 1pt,color=xfqqff] (7.52032245213483,5.585855225214912)-- (9.273969956960968,3.1313168303969534);
			\draw [line width=2.pt,dash pattern=on 1pt off 1pt,color=xfqqff] (9.273969956960968,3.1313168303969534)-- (9.249959120848901,0.11478541458712677);
			\draw [line width=2.pt,dash pattern=on 1pt off 1pt,color=xfqqff] (9.249959120848901,0.11478541458712677)-- (7.4574612670951375,-2.3115265495070623);
			\draw [line width=2.pt,dash pattern=on 1pt off 1pt,color=xfqqff] (7.4574612670951375,-2.3115265495070623)-- (4.581149651072372,-3.2208503589121458);
			\draw [line width=2.pt,dash pattern=on 1pt off 1pt,color=xfqqff] (4.581149651072372,-3.2208503589121458)-- (1.7196775478651707,-2.2658552252149122);
			\draw [line width=2.pt,dash pattern=on 1pt off 1pt,color=xfqqff] (1.7196775478651707,-2.2658552252149122)-- (-0.03396995696096806,0.18868316960304493);
			\draw [line width=2.pt,dash pattern=on 1pt off 1pt,color=xfqqff] (-0.03396995696096806,0.18868316960304493)-- (-0.009959120848901994,3.2052145854128753);
			\draw [line width=2.pt,dash pattern=on 1pt off 1pt,color=xfqqff] (-0.009959120848901994,3.2052145854128753)-- (1.782538732904862,5.63152654950706);
			\draw [line width=2.pt,dash pattern=on 1pt off 1pt,color=xfqqff] (1.782538732904862,5.63152654950706)-- (4.6588503489276265,6.540850358912145);
			\draw [line width=2.pt,dash pattern=on 1pt off 1pt,color=qqffff] (4.62,1.66)-- (4.6588503489276265,6.540850358912145);
			\draw [line width=2.pt,dash pattern=on 1pt off 1pt,color=qqffff] (7.52032245213483,5.585855225214912)-- (4.62,1.66);
			\draw [line width=2.pt,dash pattern=on 1pt off 1pt,color=qqffff] (4.62,1.66)-- (9.273969956960968,3.1313168303969534);
			\begin{scriptsize}
			\draw [fill=ududff] (4.62,1.66) circle (2.5pt);
			\draw[color=ududff] (4.399819162087391,1.8773497834163224) node {$P_{0}$};
			\draw [fill=xdxdff] (6.089586400531228,6.063352792063529) circle (2.5pt);
			\draw[color=xdxdff] (6.156906288289226,6.281886570621793) node {$P_{1}$};
			\draw [fill=xdxdff] (3.220694540916245,6.086188454209604) circle (2.5pt);
			\draw[color=xdxdff] (3.139128847323744,6.313403828490883) node {$P_n$};
			\draw [fill=xdxdff] (8.3971462045479,4.358586027805933) circle (2.5pt);
			\draw[color=xdxdff] (8.504941999536518,4.524799444419967) node {$P_2$};
			\draw [fill=uuuuuu] (4.6588503489276265,6.540850358912145) circle (2.0pt);
			\draw[color=uuuuuu] (4.801664199918304,6.707369551854521) node {$Q_n$};
			\draw [fill=uuuuuu] (7.52032245213483,5.585855225214912) circle (2.0pt);
			\draw[color=uuuuuu] (7.598820835800147,5.738213872379972) node {$Q_1$};
			\draw [fill=uuuuuu] (9.273969956960968,3.1313168303969534) circle (2.0pt);
			\draw[color=uuuuuu] (9.387425219871071,3.2247125573199633) node {$Q_2$};
			\end{scriptsize}
			\end{tikzpicture}
			}
			\caption{Inscribed and Circumscribing regular $n$-gons}
		\end{figure}

		Since $P$ and $Q$ are both regular $n$-gons, they are similar, thus, the ratios of their corresponding sides are proportional, this implies that each of their corresponding interior triangles are similar to one another as well, thus the ratios between their respective heights are proportional.
		Thus, $$\frac{h_{i}}{\eta_{i}} = \frac{s_{i}}{\sigma_{i}}.$$
		Now we get to the finish, and show that the ratio between $\A(P)$ and $\A(V_{i})$ is equal to the ratio between $s_{i}^{2}$ and $\sigma_{i}^{2}$:
			\begin{align*}
			\frac{\A(P)}{\A(Q)} &= \frac{n \cdot \tfrac{1}{2}s_{i}h_{i}}{n \cdot \tfrac{1}{2}\sigma_{i}\eta_{i}}, \\
			&= \frac{s_{i} h_{i}}{\sigma_{i} \eta_{i}}, \\
			&= \frac{s_{i}^{2}}{\sigma_{i}^{2}}.
			\end{align*}
		Thus, we have shown that the ratio between $\A(P)$ and $\A(V_{i})$ is equal to $s_{i}^{2}/\sigma_{i}^{2}$, but we can use similarity again, and conclude that $$\frac{\A(P)}{\A(Q)} = \frac{r_{i}^{2}}{\rho_{i}^{2}}.$$
		\end{proof}

		This sets us up for the big theorem, where instead of the polygons from the preceding lemma, we consider the ratio of the areas of circles, leading us to $\pi$.
		We will show that the ratio between the area of a circle and the square of its radius is constant.
		This is the Archimedean proof that uses his famous double \textit{reductio ad absurdum}.

		\begin{thm}
		If $C_{1}$ and $C_{2}$ are circles with radii $r_{1}$ and $r_{2}$, respectively, then $$\frac{\A(C_{1})}{\A(C_{2})} = \frac{r_{1}^{2}}{r_{2}^{2}},$$ or, equivalently: $$\frac{\A(C_{1})}{r_{1}^{2}} = \frac{\A(C_{2})}{r_{2}^{2}}.$$
		\end{thm}
		\begin{proof}
		Let $C_{1}, C_{2}$ be as above.
		Consider $\A(C_{1})/\A(C_{2})$, this ratio is either greater than, equal to, or less than the ratio $r_{1}^{2}/r_{2}^{2}$.
		We proceed by double \textit{reductio ad absurdum} to show $\A(C_{1})/\A(C_{2}) = r_{1}^{2}/r_{2}^{2}$.

		Suppose by way of contradiction that 
			\begin{align*}
				\frac{\A(C_{1})}{\A(C_{2})} &< \frac{r_{1}^{2}}{r_{2}^{2}}, \\
				\A(C_{1}) &< \frac{\A(C_{2})r_{1}^{2}}{r_{2}^{2}} = T.
			\end{align*}
		Let $\ep = \A(C_{2}) - T$.
		By Lemma 3.6, there exists an inscribed regular $n$-gon $P_{2}$ in $C_{2}$ such that:
			\begin{align*}
			\A(C_{2}) - \A(P_{2}) &< \ep, \\
			\A(C_{2}) - \A(P_{2}) &< \A(C_{2}) - T, \\
			\A(P_{2}) &> T.
			\end{align*}
		Using Lemma 3.6 again, there exists a regular $n$-gon inscribed $P_{1}$ in $C_{1}$ such that $P_{1} \sim P_{2}$, and by Lemma 3.7 we obtain \[\frac{\A(P_{1})}{\A(P_{2})} = \frac{r_{1}^{2}}{r_{2}^{2}}.\]
		This leads us to the first contradiction:
			\begin{align*}
				\frac{\A(P_{1})}{\A(P_{2})} &= \frac{T}{\A(P_{2})}, \\
				&= \frac{\left( \tfrac{\A(C_{1})\A(P_{2}))}{\A(P_{1})} \right)}{\A(P_{2})}, \\
				&= \frac{\A(C_{1})}{\A(P_{1})} > 1.
			\end{align*}
		This implies the following $(T/\A(P_{2})) > 1 \iff T > \A(P_{2}))$, which contradicts the above.
		Mutatis mutandis for the case where $\A(C_{1})/\A(C_{2}) > r_{1}^{2}/r_{2}^{2}$ to arrive at a similar contradiction.
		Hence, $\A(C_{1})/\A(C_{2}) = r_{1}^{2}/r_{2}^{2}$, more useful though is the expression $$ \frac{\A(C_{1})}{r_{1}^{2}} = \frac{\A(C_{2})}{r_{2}^{2}}$$ which shows us that this ratio is constant for \textit{any} circle.			
		\end{proof}

		This tells us that the ratio of the area of a circle to the square of its radius is constant, and that that constant is the same for any circle.
		Pair this knowledge with the tool of Lemma 3.6, and we can play the inscribing/circumscribing game in order to approximate this constant.
		Nowadays, we call this constant $\pi$, and we demonstrate its estimation in the Conclusion of this paper.
		% subsection approximation_of_pi (end)
	% section applications (end)

	\section{Descendants}
	\label{sec:descendants}
	In c. 212 BC, during a Roman siege on the walled city of Syracuse, Archimedes was slain in the street by a Roman footsoldier.
	With Archimedes' death (and the rise of the Roman empire) came the virtual end of Western advances in mathematics until the Italian algebraists of the 16th century.\textbf{[3]}
	Across the globe, it was not until Fermat, Newton, and Leibniz in the 16th and 17th centuries that mathematicians returned to the subject of the infinitely-small and the infinitely-large as serious concepts worth studying.
	The basic concept of the integral, that of inscribing infinitely many rectangles into a figure as a way of approximating its area, was developed by mathematicians who had closely studied Euclid, Archimedes, and everything Ancient Greek.
	Its plain to see the influence of Archimedes' basic methods in the works of future mathematicians.
	But the general idea for the integral is far from where Archimedes' influence stops.

		\subsection{The Archimedean Property}
		\label{sub:the_archimedean_property}
		Also from the field of Analysis, we have the Archimedean Property of the Real Numbers; from \textbf{[1]} we have the following statement of this property.
		The proof for this theorem relies on topological properties of $\R$, and is outside the scope of this paper, so it is omitted.
		\begin{thm}[Archimedean Property]
		\
			\begin{enumerate}
				\item Given any number $x \in \R$, there exists an $n \in \N$ satisfying $n > x$.
				\item Given any real number $y > 0$, there exists an $n \in \N$ satisfying $1/n < y$.
			\end{enumerate}
		\end{thm}
		
		What this property states is that there is no largest natural number, i.e., the natural numbers are not bounded above; and that there is no smallest real number, i.e., if you give me a number, I can always find another number smaller than yours.
		While Archimedes never concerned himself with the completeness of the real numbers, when studying $\R$ we find echoes of Archimedes' \textit{Method}, centuries after it was created.
		And this is not merely a property of the reals that we show only because of its ties to notable historical mathematical topics, even though it has no real applicable value.
		The Archimedean Property is invoked frequently in proofs related to the convergence of sequences and series, and limits of sequences and functions.\textbf{[1]}
		% subsection the_archimedean_property (end)

		\subsection{Methods of Numerical Integration}
		\label{sub:methods_of_numerical_integration}
		If we look to the field of Numerical Analysis, we find echoes of Archimedes' strategy to estimate area with inscribing polygons by \textit{The Method}.
		In the first weeks of a course on integral calculus, students learn that certain integrals, such as $\int e^{-x^{2}} \, dx$ are impossible to compute by anti-derivatives a.k.a., by hand.
		That is where Numerical Integration steps in with tools that bear a striking resemblence to Archimedes' strategy of estimating area by inscribing knowable shapes into unknowable shapes; they are the Trapezoidal Rule, and Simpson's Rule.
		The fact that we have these methods is particularly important, given their ability to evaluate tables of transformations of $\int e^{-x^{2}} \, dx$, given its importance in the field of proability theory.

		The Trapezoidal Rule allows a mathematician to inscribe the curve of a given integrand with trapezoids of uniform base length, so as to approximate the area under the curve over a given interval.
		Simpson's Rule goes a step further by inscribing a curve with uniformly spaces parabolas with negative curvature in order to approximate the definite integral of the curve.
		Both of these methods are modern flavors of Archimedes' original idea.\textbf{[4]}
		% subsection methods_of_numerical_integration (end)
	% section descendants (end)

	\section{Conclusion}
	\label{sec:conclusion}
	Throughout, we have seen how nearly two millennia before Fermat, Newton, and Leibniz, there was Archimedes of Syracuse, standing on the cusp of calculus.
	His \textit{Method} planted the seed of an idea that would bear fruit across centuries and civilizations, enabling countless mathematical advances.
	In fact, Archimedes' \textit{Exhaustive Method} of estimating $\pi$ was the best tool for such estimations until the 17th century and the advent of the infinite-series.\textbf{[3]}
	We can pair this ancient algorithm with modern software, and easily determine how long it would take to achieve certain levels of accuracy with Archimedes' tool.
	Utilizing formulae for the area of regular $n$-gons inscribed in and circumscribed about a unit circle, we use Sage to compute the following estimates of $\pi$ to twenty decimal places.\footnote{The area formulae and sage code is Appendix \ref{sec:sage_code}.}
	We start with familiar shapes like the square, hexagon, and octagon.
	From there we take off until we achieve twenty places of accuracy.
		\begin{figure}[h]
			\begin{tabular}{r|l}
				$n$ & Estimated value for $\pi$ \\
				\hline
				4 & 3.0000000000000000000 \\
				6 & 3.0310889132455352637 \\
				8 & 3.0710678118654752440 \\
				10 & 3.0940616118957144537 \\
				100 & 3.1410762904003917928 \\
				1000 & 3.1415874859178158755 \\
				1,000 & 3.1415926019126695182 \\
				$\vdots$ & $\vdots$ \\
				10,000,000,000 & 3.1415926535897932384 \\
				$\vdots$ & $\vdots$ \\
				$\infty$ & 3.1415926535897932385\dots 
			\end{tabular}
		\end{figure}

	So, with a pair of regular 10,000,000,000-gons, you could use a two-thousand year old mathematical tool to get a 20-digit approximation of $\pi$.

	We can take many lessons from Archimedes and his \textit{Method}, but we'll highlight two of them that appear in contrast to one another.
	The first is to demystify the Calculus, and Analysis generally.
	Nonmathematicians often regard Calculus as impossibly difficult for most to grasp, and a relatively new concept.
	Contrary to these assumptions, Calculus is relatively simple, at least conceptually. 
	The rough ideas behind Calculus can be explained to the nonmathematician in about twenty minutes, it is in the finer details where the difficulties lay waiting.
	And it is far from new, given that its inventors predate the United States, and that Archimedes dipped his toe into the subject before the existance of the Roman Empire.
	
	The second lesson is to venerate those who pioneered in the study of Analysis.
	Archimedes was the first mathematician to approach the concepts that would form the bedrock of Analysis, and that was with a unique pairing of supreme intellect and prolific work-ethic that would go unrivaled in the Western world for centuries.
	It took the cleverness and hard-work of the greatest mathematicians of the Rennaissance and Enlightenment periods to pickup where Archimedes left off and invent a new subject of mathematics that we know call Analysis.
	And in modern times, it is often a university student's first course in real analysis that introduces them to what it actually means for mathematics to be difficult.
	That Archimedes was able to get so close to such a difficult topic with only the tools of Euclidean Geometry is truly astounding.
	% section conclusion (end)
	\newpage

	\appendix

	\section{References}
	\label{sec:references}
	\begin{enumerate}
		\item Abbott, Stephen; \textit{Understanding Analysis}, 2nd edition
		\item Archimedes; \textit{The Works of Archimedes}
		\item Beckmann, Petr; \textit{A History of $\pi$}
		\item Cheney, Ward \& Kincaid, David; \textit{Numerical Mathematics and Computing}, 6th edition
		\item Dunham, William; \textit{Journey Through Genius}
		\item Edwards Jr., C.H.; \textit{The Historical Development of the Calculus}
		\item Euclid; \textit{The Elements}
	\end{enumerate}
	% section references (end)

	\section{Sage Code}
	\label{sec:sage_code}
	I'm not great at using Sage yet, so I'm certain that there exists a cleaner, more elegant way to write this code as a function, so as to not change $n$ and $m$ by hand so much.

	These use the following area formulae, found in my class notes from Math 395.
	Let $C$ be a unit circle with regular $n$-gon $P$ inscribed, and circumscribed by the regular $n$-gon $Q$, then \[ \A(P) = \left(\frac{n}{2} \right) \sin(\alpha), \hspace{2.5mm} \text{and} \hspace{2.5mm} \A(Q) = \left(\frac{n}{2} \right)\sin(\alpha)\left(\frac{1}{\cos(\tfrac{\alpha}{2})}\right)^{2}. \]

	\begin{verbatim}
		n = var('n')
		alpha = var('alpha')

		insc(alpha,n) = 0.5*n*sin(alpha)
		circ(alpha,n) = (n/2)*sin(alpha)*((1/cos((alpha/2)))^2)

		n = 10000000000
		alpha = (2*pi)/n

		A = N(insc(alpha,n), digits=20)
		B = N(circ(alpha,n), digits=20)
		C = N((A+B)/2,digits=20)
		P = N(pi,digits=20)

		print 'Estimated value of pi =', C
		print 'Actual value of pi    =', P
	\end{verbatim}
	% section sage_code (end)
\end{document}