%!TEX output_directory = temp
\documentclass[letterpaper, 12pt]{amsart}
	%%%%%%%%%%%%%%%%%%%%%%%%%%%%%%%%%%%%%%%%%%%%%%%%%%%%%%%%%%%%%%%%%%%%%%%%%%%%%%
	%%%%%%%%%%%%%%%%%%%%%%%%%%%% boilerplate packages %%%%%%%%%%%%%%%%%%%%%%%%%%%%
		\usepackage[margin=1.25in]{geometry}
		\usepackage{amsmath,amssymb,amsthm,bm}
		\usepackage{marvosym}
		\usepackage[mathscr]{euscript}
		\usepackage{enumerate}
		\usepackage{graphicx}
		\usepackage{mathrsfs}
		\usepackage{color}
		\usepackage{hyperref}
		\usepackage{verbatim}
		\usepackage{stmaryrd}
		\usepackage{tikz}
		\usetikzlibrary{arrows}

	%%%%%%%%%%%%%%%%%%%%%%%%%%%%%%%%%%%%%%%%%%%%%%%%%%%%%%%%%%%%%%%%%%%%%%%%%%%%%%
	%%%%%%%%%%%%%%%%%%%%%%%%%%%%% rename the abstract %%%%%%%%%%%%%%%%%%%%%%%%%%%%
		% \renewcommand{\abstractname}{Introduction}

	%%%%%%%%%%%%%%%%%%%%%%%%%%%%%%%%%%%%%%%%%%%%%%%%%%%%%%%%%%%%%%%%%%%%%%%%%%%%%%
	%%%%%%%%%%%%%%%%%%%%%%%%%%%%%%%%%%%%% sets %%%%%%%%%%%%%%%%%%%%%%%%%%%%%%%%%%%
		\DeclareMathOperator{\N}{\mathbb{N}}				% natural numbers
		\DeclareMathOperator{\Z}{\mathbb{Z}}				% integers
		\DeclareMathOperator{\Zp}{\mathbb{Z}^{+}}			% positive integers
		\DeclareMathOperator{\Q}{\mathbb{Q}}				% rationals
		\DeclareMathOperator{\Qc}{\mathbb{Q}^{c}}			% irrationals
		\DeclareMathOperator{\R}{\mathbb{R}}				% reals
		\DeclareMathOperator{\F}{\mathbb{F}}				% a field
		\DeclareMathOperator{\C}{\mathbb{C}}				% complex numbers
		\DeclareMathOperator{\Cnon}{\mathbb{C}^{\times}}	% nonzero complex numbers
		\DeclareMathOperator{\Pcal}{\mathcal{P}}			% powerset, or set of polynomials
		\DeclareMathOperator{\Ell}{\mathscr{L}}				% set of linear maps, or linear operator

	%%%%%%%%%%%%%%%%%%%%%%%%%%%%%%%%%%%%%%%%%%%%%%%%%%%%%%%%%%%%%%%%%%%%%%%%%%%%%%
	%%%%%%%%%%%%%%%%%%%%%%%%%%%%%% use pretty letters %%%%%%%%%%%%%%%%%%%%%%%%%%%%
		\DeclareMathOperator{\ep}{\varepsilon}				% epsilons
		\DeclareMathOperator{\ph}{\varphi}					% phis

	%%%%%%%%%%%%%%%%%%%%%%%%%%%%%%%%%%%%%%%%%%%%%%%%%%%%%%%%%%%%%%%%%%%%%%%%%%%%%%
	%%%%%%%%%%%%%%%%%%%%%%%%%%%%%%%%%%% algebra %%%%%%%%%%%%%%%%%%%%%%%%%%%%%%%%%%
		\renewcommand{\null}{\text{null }}					% null space
		\DeclareMathOperator{\range}{\text{range }}			% range
		\newcommand{\bmat}[1]{{\mathbf{#1}}}				% bold matrix
		\newcommand{\bvec}[1]{{\vec{\mathbf{#1}}}}			% bold vector
		\DeclareMathOperator{\ind}{\perp\!\!\!\perp}		% perpendicular, orthogonal
		\DeclareMathOperator{\ord}{\text{ord}}				% order of a structure
		\DeclareMathOperator{\Log}{Log}						% logarithm
		\DeclareMathOperator{\Span}{Span}					% span
		\newcommand{\pid}[1]{\langle #1 \rangle}			% bracket notation, used for 
															% ideals or inner products
		\newcommand{\norm}[1]{\mid \!\!#1 \!\!\mid}			%\norm{x} gives |x|

		% fatdot notation
		\makeatletter
			\newcommand*\fatdot{\mathpalette\fatdot@{.5}}
			\newcommand*\fatdot@[2]{\mathbin{\vcenter{\hbox{\scalebox{#2}{$\m@th#1\bullet$}}}}}
		\makeatother

	%%%%%%%%%%%%%%%%%%%%%%%%%%%%%%%%%%%%%%%%%%%%%%%%%%%%%%%%%%%%%%%%%%%%%%%%%%%%%%
	%%%%%%%%%%%%%%%%%%%%%%%%%%% probability & statistics %%%%%%%%%%%%%%%%%%%%%%%%%
		\renewcommand{\Pr}{\mathbb{P}}						% probability
		\DeclareMathOperator{\E}{\mathbb{E}}				% expectation
		\DeclareMathOperator{\var}{\rm Var}					% variance
		\DeclareMathOperator{\sd}{\rm SD}					% standard deviation
		\DeclareMathOperator{\cov}{\rm Cov}					% covariance
		\DeclareMathOperator{\SE}{\rm SE}					% standard error
		\DeclareMathOperator{\ssreg}{{\rm SS}_{{\rm Reg}}}	% sum of squared regression
		\DeclareMathOperator{\ssr}{{\rm SS}_{{\rm Res}}}	% sum of squared residuals
		\DeclareMathOperator{\sst}{{\rm SS}_{{\rm Tot}}}	% total sum of squares

	%%%%%%%%%%%%%%%%%%%%%%%%%%%%%%%%%%%%%%%%%%%%%%%%%%%%%%%%%%%%%%%%%%%%%%%%%%%%%%
	%%%%%%%%%%%%%%%%%%%%%%%%%%%%%%% number theory %%%%%%%%%%%%%%%%%%%%%%%%%%%%%%%%
		\renewcommand{\mod}[1]{\ (\mathrm{mod}\ #1)}		% congruences

	%%%%%%%%%%%%%%%%%%%%%%%%%%%%%%%%%%%%%%%%%%%%%%%%%%%%%%%%%%%%%%%%%%%%%%%%%%%%%%
	%%%%%%%%%%%%%%%%%%%%%%%%%%%% theorem environments %%%%%%%%%%%%%%%%%%%%%%%%%%%%
		% Some theorem-like environments, all numbered together starting at 1
		% in each section.

		\newtheorem{thm}{Theorem}[section]					% The default \theoremstyle is 
		\newtheorem{defn}[thm]{Definition}					% bold headings and italic body text.
		\newtheorem{prop}[thm]{Proposition}
		\newtheorem{claim}[thm]{Claim}
		\newtheorem{cor}[thm]{Corollary}
		\newtheorem{lemma}[thm]{Lemma}

		\theoremstyle{definition}  							% Bold headings and Roman body text.
		\newtheorem{example}[thm]{Example}
		\newtheorem{examples}[thm]{Examples}
		\newtheorem{exercise}[thm]{Exercise}
		\newtheorem{note}[thm]{Note}
		\newtheorem{remark}[thm]{Remark}
		\newtheorem{remarks}[thm]{Remarks}
		\newtheorem{discussion}[thm]{Discussion}

		\newcommand{\dfn}{\textbf} 							% Make defined words bold.
		\newcommand{\mdfn}[1]{\dfn{\mathversion{bold}#1}} 	% Even make math symbols bold

	%%%%%%%%%%%%%%%%%%%%%%%%%%%%%%%%%%%%%%%%%%%%%%%%%%%%%%%%%%%%%%%%%%%%%%%%%%%%%%
	%%%%%%%%%%%%%%%%%%%%%%%%%%%%%%% complex numbers %%%%%%%%%%%%%%%%%%%%%%%%%%%%%%
		\DeclareMathOperator{\Arg}{Arg}						% argument of z \in \C
		\DeclareMathOperator{\re}{Re}						% real component
		\DeclareMathOperator{\im}{Im}						% imaginary component

	%%%%%%%%%%%%%%%%%%%%%%%%%%%%%%%%%%%%%%%%%%%%%%%%%%%%%%%%%%%%%%%%%%%%%%%%%%%%%%
	%%%%%%%%%%%%%%%%%%%%%%%%%%%%%%% various symbols %%%%%%%%%%%%%%%%%%%%%%%%%%%%%%
		\newcommand{\iso}{\cong}						% isometric/congruent
		\newcommand{\ra}{\rightarrow}                   % right arrow
		\newcommand{\Ra}{\Rightarrow}                   % right implies
		\newcommand{\lra}{\longrightarrow}              % long right arrow
		\newcommand{\la}{\leftarrow}                    % left arrow
		\newcommand{\La}{\Leftarrow}                    % left implies
		\newcommand{\lla}{\longleftarrow}               % long left arrow
		\newcommand{\eqra}{\llra{\sim}}                 % equivalence/isomorphism
		\newcommand{\blank}{\underbar{\ \ }}          	% An underscore, as in (__)xV
		% \newcommand{\blank}{-}                          % A hyphen, as in (-)xV
		\newcommand{\Id}{Id}                            % The identity functor
		\newcommand{\und}{\underline}
		\newcommand{\del}{\nabla}						% gradient vector

		\raggedbottom	
\begin{document}
	\title{[TITLE]}
	\author{Alex Thies \\ \lowercase{athies@uoregon.edu}}

	\pagenumbering{gobble}
	\maketitle
	\newpage

	\pagenumbering{roman}
	\tableofcontents
	\newpage

	\section*{Notes on Notation}
	\label{sec:notes_on_notation}
	Vectors will be bold with an arrow on top, i.e., $\bvec{v}$ is a vector.
	All vectors herein will be elements of the vector space $\R^3$.
	% section notes_on_notation (end)
	\newpage

	\pagenumbering{arabic}
	\section{Introduction}
		\subsection{Copernicus, Kepler, and Newton}
		\label{sub:copernicus_kepler_and_newton}
		% subsection copernicus_kepler_and_newton (end)

		\subsection{Neptune \& Halley's Comet}
		\label{sub:neptune_&_halley_s_comet}
		% subsection neptune_&_halley_s_comet (end)

		\subsection{Einstein and Space-time}
		\label{sub:einstein_and_space_time}
		% subsection einstein_and_space_time (end)
	\label{sec:introduction}
	% section introduction (end)
	\newpage

	\section{Fundamentals of Orbital Mechanics}
	\label{sec:fundamentals_of_orbital_mechanics}
		First we have to start with some basic definitions.

		\begin{defn}[Distance]
			The Euclidean distance formula for points $(x_{1},y_{1},z_{1}),(x_{2},y_{2},z_{2}) \in \R^3$ is $$d((x_{1},y_{1},z_{1}),(x_{2},y_{2},z_{2})) = \sqrt{(x_{1} - x_{2})^2 + (y_{1} - y_{2})^2 + (z_{1} - z_{2})^2}.$$
			Note, this also works for points in $\R^2$, where $z_{1} = z_{2} = 0$.
		\end{defn}

		\begin{defn}[Circle]
			Let $x,y,r,h,k \in \R$.
			Then, we say that the set of points $(x,y)$ of equal distance $r$ from a point $(h,k)$ is a \textbf{circle}, i.e., $$C = \left\{ (x,y) : (x-h)^2 + (y-k)^2 = r^2 \text{ for $x,y,r \in \R$}\right\}$$ 
		\end{defn}

		\begin{defn}[Ellipse]
			Let $x,y,a,b,h,k \in \R$.
			Without loss of generality assume $b \leq a$.
			Then, we say that the set $$E = \left\{ (x,y) : \tfrac{(x-h)^2}{a^2} + \tfrac{(y-k)^2}{b^2} = 1 \right\}$$ is an \textbf{ellipse} centered about the point $(h,k)$ whose \textbf{major axis} is of length $b$, and whose \textbf{minor axis} is of length $a$.
			The \textbf{eccentricity} of an ellipse is defined as $$e = \sqrt{1 - \tfrac{b^2}{a^2}},$$ this helps us define the foci of an ellipse.
			The two \textbf{foci} of an ellipse are points lying on the major axis, they are $F_{1} := (-ae,0)$ and $F_{2} := (ae,0)$.
			With these characters we can also define an ellipse thus, $$E = \{ (x,y) : d((x,y),(-ae,0)) + d((x,y),(ae,0)) = 2a \}$$ where $d$ is the Euclidean distance.
		\end{defn}
	
		\subsection{Newton's Laws}
		\label{sub:newtons_laws}
			\begin{thm}[Newton's 1st Law of Motion]
			Every body persists in its state of being at rest or of moving uniformly straight forward, except insofar as it is compelled to change its state by force impressed.
			\end{thm}
			This describes what we call inertia, or uniform motion.
			This statement has been refined over time, a modern statement of Newton's 1st Law is along the lines of: ``If the sum of force vectors acting on an object is zero, then and only then is the velocity of the object is zero.''
			We can express this mathematically as $$\sum \bvec{F} = 0 \iff \frac{d\bvec{v}}{dt} = 0.$$

			\begin{proof}
			\end{proof}

			\begin{thm}[Newton's 2nd Law of Motion]
			The change of momentum of a body is proportional to the impulse impressed on the body, and happens along the straight line on which that impulse is impressed.
			\end{thm}
			Oftentimes this law is stated as ``force equals the product of mass and acceleration,'' however this is more of a consequence of the law, rather than a statement of the law itself.
			Recall that momentum is defined as $\vec{\bm{\rho}} = m \bvec{v}$, where $m$ is a constant. 
			Therefore in the language of differential calculus, we can describe the change of momentum of a body thus,
				\begin{align*}
					\bvec{F} &= \frac{d\vec{\bm{\rho}}}{dt}, \\
					&= \frac{d(m\bvec{v})}{dt}, \\
					&= m\frac{d\bvec{v}}{dt}, \\
					&= m \bvec{a}.
				\end{align*}
			This ability to express the force on an object as a derivative is very useful when it comes to differential equations.

			\begin{proof}
			\end{proof}

			\begin{thm}[Newton's 3rd Law of Motion]
			To every action there is always opposed an equal reaction: or the mutual actions of two bodies upon each other are always equal, and directed to contrary parts.
			\end{thm}
			Most people know this law as ``every action has an equal and opposite reaction.''

			\begin{proof}
			\end{proof}

			\begin{thm}[Newton's Law of Universal Gravitation]
			Every point mass attracts every single other point mass by a force acting along the line intersecting both points. 
			The force is proportional to the product of the two masses and inversely proportional to the square of the distance between them, $$\bvec{F}_{g} = \frac{GMm}{r^2}$$ where $\bvec{F}_{g}$ is the force of gravity; $G = 6.674 \times 10^{-11} \, m^3/kg \, s^2$ is the Gravitational constant; $M$ is the mass of larger object, $m$ the mass of the smaller one, and $r$ is the distance between them.
			\end{thm}

			\begin{proof}
			\end{proof}
		% subsection newtons_laws (end)

		\subsection{Kepler's Laws of Planetary Motion}
		\label{sub:kepler_s_laws_of_planetary_motion}
			\begin{thm}[Kepler's 1st Law]
			The orbit of a planet is an ellipse with the Sun at one of the two foci.
			\end{thm}

			\begin{thm}[Kepler's 2nd Law]
			A line segment joining a planet and the Sun sweeps out equal areas during equal intervals of time.
			\end{thm}

			\begin{thm}[Kepler's 3rd Law]
			The square of the orbital period of a planet is proportional to the cube of the semi-major axis of its orbit.
			\end{thm}

			\begin{proof}[Proof of all Three Laws]
			We begin the proof by considering circular motion.
			Let $C$ be a circle of radius $r$, and a particle $P$ lying on the boundary of $C$.
			Let $T$ be the time required for $P$ to complete a full rotation around the bounadry of $C$ starting from a fixed point.
			Then, the velocity $v$ of $P$ is $v = \tfrac{2\pi r}{T}$, and the acceleration $a = \tfrac{2\pi v}{T}$.
			Solve each of these for $T$, equate them, and then solve for $a$.
				\begin{align*}
					\frac{2\pi r}{v} &= \frac{2\pi v}{a}, \\
					ar &= v^2, \\
					a &= \frac{v^2}{r}.
				\end{align*}
			With $a$ in terms of $v$, we can invoke Newton's Second Law, which states that the forces acting on $P$ can be written as $\bvec{F}_{p} = ma$.
			With $a = v^2/r$, we have $$\bvec{F}_{p} = m \frac{v^2}{r}.$$
			If we think of $v$ and $a$ as vectors, then $\bvec{v}$ is pointed in the direction of motion, tangent to $C$, and $\bvec{a}$ is pointed towards the center of $C$, i.e., $\bvec{v} \perp \bvec{a}$.
			Therefore we can decompose the forces acting on $P$ into orthogonal components.

			Our next goal is to generalize this theory to ellipses, and again decompose the forces acting on a particle travelling along an ellipse into orthogonal components.
			The ideal coordinate system for this is polar coordinates on the complex plane $\C^2$.
			Let $Q$ be a particle traveling along the boundary of an ellipse that is centered about the origin.
			In polar coordinates over $\C^2$, we can write $Q = r e^{i\theta} = \cos(\theta) + i \sin(\theta)$.
			Thus, the position of $Q$ is a function in two variables, call it $s(r,\theta)$.
			As with the case of circular motion, we need to find ways of expressing the velocity and accleration of $Q$.
			In this case, we do that by taking derivatives of $s$ with respect to time $t$.
			We have
				\begin{align*}
					\frac{ds}{dt} &= \frac{dr}{dt}e^{i\theta} + r\frac{d\theta}{dt}e^{i(\theta + \pi/2)}; \\
					\frac{d^2s}{dt^2} &= \left( \frac{d^2r}{dt^2} - r\left(\frac{d\theta}{dt}\right)^2 \right)e^{i\theta} + \left( 2\frac{dr}{dt}\frac{d\theta}{dt} + r\frac{d^2\theta}{dt^2} \right)e^{i(\theta + \pi/2)}; \\
				\end{align*}
			
			\end{proof}
		% subsection kepler_s_laws_of_planetary_motion (end)
	% section fundamentals_of_orbital_mechanics (end)
	\newpage

	\section{The $n$-body Problem}
	\label{sec:the_n-body_problem}
	% section the_n-body_problem (end)
	\newpage

	\section{Conclusion}
	\label{sec:conclusion}
	% section conclusion (end)

	\newpage
	\appendix

	\section{References}
	\label{sec:references}
	% section references (end)
\end{document}